%!TEX TS-program = xelatex
\documentclass{article}
\usepackage{amsmath}
\usepackage{amssymb}
\usepackage{fvextra}
\usepackage{tcolorbox}
\usepackage{listings}
\usepackage{amsthm}
\usepackage{fontspec}  % For Unicode support
\setmonofont{DejaVu Sans Mono}  % For monospace/code blocks
\usepackage{unicode-math}
% \setmathfont{XITS Math}  % Or another Unicode math font
\newtheorem{example}{Example}
\fvset{breaklines=true}



\begin{document}
\lstdefinelanguage{Lean4}{
    breaklines=true,
    basicstyle=\ttfamily\normalsize,
    keepspaces=true,
    morekeywords={
        def, theorem, lemma, example, have, show, calc, let, assume, by, exact,
        sorry, obvious, Type, Prop, where, with, extends, class, instance,
        structure, inductive, mutual, coinductive, variable, variables,
        universe, universes, deriving, abbrev, partial, terminating,
        namespace, import, open, export, private, protected, public,
        noncomputable, unsafe, macro, syntax, macro_rules, set_option,
        attribute, local, scoped, section, end, match, fun, if, then, else,
        return, do, for, in, while, break, continue, try, catch, throw,
        mut, pure, opaque
    },
    morekeywords=[2]{
        ℕ, ℝ, ∈, ≥, ≤, →, ∀, ∃, ⊢, ∧, ∨, ¬, ≠, ×, ⊗, ⊕, ∘, □, ◇, ∎,
        ⟨, ⟩, ⦃, ⦄, ▸, ≈, ∼, ≡, ⌊, ⌋, ⌈, ⌉
    },
    morecomment=[l]--,     % Line comments start with --
    morestring=[b]",       % Strings in double quotes
    sensitive=true,         % Case-sensitive
    keywordstyle=\color{blue}\bfseries,      % Regular keywords in blue and bold
    keywordstyle=[2]\color{purple}\bfseries, % Special symbols in purple and bold
    commentstyle=\color{green!50!black},     % Comments in dark green
    stringstyle=\color{red},                 % Strings in red
}



\begin{example}
Problem:
\begin{tcolorbox}[colback=yellow!10, width=\linewidth]
Prove that for all real numbers $a$ and $b$:
    $$(a+b)^2 \geq 0.$$
\end{tcolorbox}

Simplified proof:
\begin{tcolorbox}[colback=blue!10, width=\linewidth]
Trivial since the square of any real number is non-negative.
\end{tcolorbox}
\end{example}

Elaborated proof:
\begin{tcolorbox}[colback=green!10, width=\linewidth]
Trivial since the square of any real number is non-negative.
\end{tcolorbox}

Regularized proof:
\begin{tcolorbox}[colback=red!10, width=\linewidth]
Let $a\in\mathbb{R}$.
Let $b\in\mathbb{R}$.
The goal is to prove ${{(a+b)}}^2 \ge 0$.
We have ${{(a+b)}}^2 \geq 0$.
\end{tcolorbox}


\begin{example}
Problem:
\begin{tcolorbox}[colback=yellow!10, width=\linewidth]
Prove that for any positive real numbers $x$ and $y$:
    $$\frac{x+y}{2} \geq \sqrt{xy}.$$
\end{tcolorbox}

Simplified proof:
\begin{tcolorbox}[colback=blue!10, width=\linewidth]
Since $x$ and $y$ are positive, $(\sqrt x - \sqrt y)^2 \ge 0$. Thus $\frac{x+y}{2} \ge \sqrt{xy}$.
\end{tcolorbox}
\end{example}

Elaborated proof:
\begin{tcolorbox}[colback=green!10, width=\linewidth]
Since $x$ and $y$ are positive, $(\sqrt x - \sqrt y)^2 = (\sqrt x)^2 - 2\sqrt x \sqrt y + (\sqrt y)^2 = x - 2\sqrt{xy} + y \ge 0$. Thus $x + y \ge 2\sqrt{xy}$, so $\frac{x+y}{2} \ge \sqrt{xy}$.
\end{tcolorbox}

Regularized proof:
\begin{tcolorbox}[colback=red!10, width=\linewidth]
Let $x\in\mathbb{R}$. Assume $x\ge 0$.
Let $y\in\mathbb{R}$. Assume $y\ge 0$.
The goal is to prove $\frac{x+y}{2} \ge \sqrt{xy}$.
We have ${{(\sqrt x - \sqrt y)}}^2 = {{(\sqrt x)}}^2 - 2\sqrt x \sqrt y + {{(\sqrt y)}}^2$.
We have ${{(\sqrt x)}}^2 - 2\sqrt x \sqrt y + {{(\sqrt y)}}^2 = x - 2\sqrt{xy} + y$.
We have $x - 2\sqrt{xy} + y \ge 0$.
We have ${{(\sqrt x - \sqrt y)}}^2 \ge 0$.
We have $x + y \ge 2\sqrt{xy}$.
We have $\frac{x+y}{2} \ge \sqrt{xy}$.
\end{tcolorbox}


\begin{example}
Problem:
\begin{tcolorbox}[colback=yellow!10, width=\linewidth]
Show that for all real numbers $a$, $b$, and $c$:
    $$a^2 + b^2 + c^2 \geq ab + bc + ca.$$
\end{tcolorbox}

Simplified proof:
\begin{tcolorbox}[colback=blue!10, width=\linewidth]
We have $2(a^2 + b^2 + c^2 - ab - bc - ca) = (a-b)^2 + (b-c)^2 + (c-a)^2 \ge 0$. Thus, $a^2 + b^2 + c^2 \geq ab + bc + ca$.
\end{tcolorbox}
\end{example}

Elaborated proof:
\begin{tcolorbox}[colback=green!10, width=\linewidth]
We have $2(a^2 + b^2 + c^2 - ab - bc - ca) = 2a^2 + 2b^2 + 2c^2 - 2ab - 2bc - 2ca = (a^2 - 2ab + b^2) + (b^2 - 2bc + c^2) + (c^2 - 2ca + a^2) = (a-b)^2 + (b-c)^2 + (c-a)^2 \ge 0$. Thus, $2(a^2 + b^2 + c^2 - ab - bc - ca) \ge 0$, so $a^2 + b^2 + c^2 - ab - bc - ca \ge 0$, and hence $a^2 + b^2 + c^2 \geq ab + bc + ca$.
\end{tcolorbox}

Regularized proof:
\begin{tcolorbox}[colback=red!10, width=\linewidth]
Let $a\in\mathbb{R}$.
Let $b\in\mathbb{R}$.
Let $c\in\mathbb{R}$.
The goal is to prove $a^2 + b^2 + c^2 \geq ab + bc + ca$.
We have $2(a^2 + b^2 + c^2 - ab - bc - ca) = 2a^2 + 2b^2 + 2c^2 - 2ab - 2bc - 2ca = (a^2 - 2ab + b^2) + (b^2 - 2bc + c^2) + (c^2 - 2ca + a^2) = {(a-b)}^2 + {(b-c)}^2 + {(c-a)}^2$.
We have ${(a-b)}^2 + {(b-c)}^2 + {(c-a)}^2 \ge 0$.
We have $2(a^2 + b^2 + c^2 - ab - bc - ca) \ge 0$.
We have $a^2 + b^2 + c^2 - ab - bc - ca \ge 0$.
We have $a^2 + b^2 + c^2 \geq ab + bc + ca$.
\end{tcolorbox}


\begin{example}
Problem:
\begin{tcolorbox}[colback=yellow!10, width=\linewidth]
Prove that for any positive real number $x$:
    $$x + \frac{1}{x} \geq 2.$$
\end{tcolorbox}

Simplified proof:
\begin{tcolorbox}[colback=blue!10, width=\linewidth]
Since $(x-1)^2 \ge 0$ for all $x$, dividing by $x$ gives $x + \frac{1}{x} - 2 \ge 0$, so $x + \frac{1}{x} \ge 2$.
\end{tcolorbox}
\end{example}

Elaborated proof:
\begin{tcolorbox}[colback=green!10, width=\linewidth]
Since $(x-1)^2 = x^2 - 2x + 1 \ge 0$ for all $x$, dividing by $x$ gives $\frac{x^2 - 2x + 1}{x} = x + \frac{1}{x} - 2 \ge 0$, so $x + \frac{1}{x} \ge 2$.
\end{tcolorbox}

Regularized proof:
\begin{tcolorbox}[colback=red!10, width=\linewidth]
Let $x\in\mathbb{R}$. Assume $x>0$.
The goal is to prove $x + \frac{1}{x} \ge 2$.
We have ${\left(x-1\right)}^2 = x^2 - 2x + 1 \ge 0$.
We have $\frac{x^2 - 2x + 1}{x} = x + \frac{1}{x} - 2 \ge 0$ by dividing by $x$.
We have $x + \frac{1}{x} \ge 2$.
\end{tcolorbox}


\begin{example}
Problem:
\begin{tcolorbox}[colback=yellow!10, width=\linewidth]
For positive real numbers $a$ and $b$, prove:
    $$\left(\frac{a+b}{2}\right)^2 \leq \frac{a^2+b^2}{2}.$$
\end{tcolorbox}

Simplified proof:
\begin{tcolorbox}[colback=blue!10, width=\linewidth]
We have
$ \frac{a^2+b^2}{2} - \left(\frac{a+b}{2}\right)^2 = \frac{(a-b)^2}{4} \ge 0. $
Thus, the inequality holds.
\end{tcolorbox}
\end{example}

Elaborated proof:
\begin{tcolorbox}[colback=green!10, width=\linewidth]
We have
$ \frac{a^2+b^2}{2} - \left(\frac{a+b}{2}\right)^2 = \frac{2a^2+2b^2}{4} - \frac{a^2+2ab+b^2}{4} = \frac{a^2-2ab+b^2}{4} = \frac{(a-b)^2}{4} \ge 0. $
Thus, the inequality holds.
\end{tcolorbox}

Regularized proof:
\begin{tcolorbox}[colback=red!10, width=\linewidth]
Let $a\in\mathbb{R}$.
Let $b\in\mathbb{R}$.
The goal is to prove $\frac{a^2+b^2}{2} \ge {\left(\frac{a+b}{2}\right)}^2$.
We have $\frac{a^2+b^2}{2} - {\left(\frac{a+b}{2}\right)}^2 = \frac{2a^2+2b^2}{4} - \frac{a^2+2ab+b^2}{4} = \frac{a^2-2ab+b^2}{4} = \frac{{{(a-b)}}^2}{4} \ge 0$.
We have $\frac{a^2+b^2}{2} - {\left(\frac{a+b}{2}\right)}^2 \ge 0$.
We have $\frac{a^2+b^2}{2} \ge {\left(\frac{a+b}{2}\right)}^2$.
\end{tcolorbox}
\end{document}