%!TEX TS-program = xelatex
\documentclass{article}
\usepackage{amsmath}
\usepackage{amssymb}
\usepackage{fvextra}
\usepackage{tcolorbox}
\usepackage{listings}
\usepackage{amsthm}
\usepackage{fontspec}  % For Unicode support
\setmonofont{DejaVu Sans Mono}  % For monospace/code blocks
\usepackage{unicode-math}
% \setmathfont{XITS Math}  % Or another Unicode math font
\newtheorem{example}{Example}
\fvset{breaklines=true}



\begin{document}
\lstdefinelanguage{Lean4}{
    breaklines=true,
    basicstyle=\ttfamily\normalsize,
    keepspaces=true,
    morekeywords={
        def, theorem, lemma, example, have, show, calc, let, assume, by, exact,
        sorry, obvious, Type, Prop, where, with, extends, class, instance,
        structure, inductive, mutual, coinductive, variable, variables,
        universe, universes, deriving, abbrev, partial, terminating,
        namespace, import, open, export, private, protected, public,
        noncomputable, unsafe, macro, syntax, macro_rules, set_option,
        attribute, local, scoped, section, end, match, fun, if, then, else,
        return, do, for, in, while, break, continue, try, catch, throw,
        mut, pure, opaque
    },
    morekeywords=[2]{
        ℕ, ℝ, ∈, ≥, ≤, →, ∀, ∃, ⊢, ∧, ∨, ¬, ≠, ×, ⊗, ⊕, ∘, □, ◇, ∎,
        ⟨, ⟩, ⦃, ⦄, ▸, ≈, ∼, ≡, ⌊, ⌋, ⌈, ⌉
    },
    morecomment=[l]--,     % Line comments start with --
    morestring=[b]",       % Strings in double quotes
    sensitive=true,         % Case-sensitive
    keywordstyle=\color{blue}\bfseries,      % Regular keywords in blue and bold
    keywordstyle=[2]\color{purple}\bfseries, % Special symbols in purple and bold
    commentstyle=\color{green!50!black},     % Comments in dark green
    stringstyle=\color{red},                 % Strings in red
}



\begin{example}
Problem:
\begin{tcolorbox}[colback=yellow!10, width=\linewidth]
Prove: $1+1=2$.
\end{tcolorbox}

Simplified proof:
\begin{tcolorbox}[colback=blue!10, width=\linewidth]
This is trivial.
\end{tcolorbox}
\end{example}

Elaborated proof:
\begin{tcolorbox}[colback=green!10, width=\linewidth]
This is trivial.
\end{tcolorbox}

Regularized proof:
\begin{tcolorbox}[colback=red!10, width=\linewidth]
The goal is to prove $1+1=2$.
We have $1+1=2$.
\end{tcolorbox}

Lean 4 code:
\begin{tcolorbox}[colback=white!10, width=\linewidth]
\begin{lstlisting}[language=Lean4]
import Mathlib
syntax "attack" : tactic

macro_rules
| `(tactic| attack) => `(tactic|
  first
  | simp; done
  | ring; done
  | ring_nf; rw [Real.sq_sqrt]; ring; all_goals attack; done
  | nlinarith; done
  | apply sq_nonneg; all_goals attack; done
  | apply div_nonneg; all_goals attack; done
  | field_simp; ring; done
  | linarith; done
)

macro "obvious": tactic =>`(tactic|
  first
  | attack; done
  | congr; all_goals attack; done
  | gcongr; all_goals attack; done
  | fail "Could not prove this goal automatically. It might not be as obvious as you think!"
)

macro "in_set" : term => `(true)


macro "term_trivial": tactic =>`(tactic|
  first
  | simp; done
  | ring; done
  | ring_nf; done
  | linarith; done
  | fail "Could not prove this goal automatically. Afterall, this is an ad hoc implementation."
)

macro "old_main_hypothesis": tactic =>`(tactic|
  first
  | assumption; done
  | fail "Could not prove this goal automatically. Afterall, this is an ad hoc implementation."
)

macro "let_assigned": tactic =>`(tactic|
  first
  | dsimp; done
  | fail "Could not prove this goal automatically. Afterall, this is an ad hoc implementation."
)

macro "term_equivalence": tactic =>`(tactic|
  first
  | simp; done
  | ring; done
  | ring_nf; done
  | linarith; done
  | fail "Could not prove this goal automatically. Afterall, this is an ad hoc implementation."
)

macro "comm_ring": tactic =>`(tactic|
  first
  | ring; done
  | ring_nf; done
  | fail "Could not prove this goal automatically. Afterall, this is an ad hoc implementation."
)

macro "litnum_reduce": tactic =>`(tactic|
  first
  | simp; done
  | simp [*]; done
  | fail "Could not prove this goal automatically. Afterall, this is an ad hoc implementation."
)

macro "litnum_bound": tactic =>`(tactic|
  first
  | linarith; done
  | fail "Could not prove this goal automatically. Afterall, this is an ad hoc implementation."
)

def h : (1:ℕ) + (1:ℕ) = (2:ℕ) := by
  have h1 : (1:ℕ) + (1:ℕ) = (2:ℕ) := by term_trivial
  obvious

\end{lstlisting}
\end{tcolorbox}


\begin{example}
Problem:
\begin{tcolorbox}[colback=yellow!10, width=\linewidth]
Let $x\in\mathbb{R}$. Prove: $x^2\ge 0$.
\end{tcolorbox}

Simplified proof:
\begin{tcolorbox}[colback=blue!10, width=\linewidth]
The square of a real number is non-negative.
\end{tcolorbox}
\end{example}

Elaborated proof:
\begin{tcolorbox}[colback=green!10, width=\linewidth]
The square of a real number is non-negative.
\end{tcolorbox}

Regularized proof:
\begin{tcolorbox}[colback=red!10, width=\linewidth]
Let $x\in\mathbb{R}$.
The goal is to prove ${{x}}^2 \ge 0$.
We have ${{x}}^2 \ge 0$.
\end{tcolorbox}

Lean 4 code:
\begin{tcolorbox}[colback=white!10, width=\linewidth]
\begin{lstlisting}[language=Lean4]
import Mathlib
syntax "attack" : tactic

macro_rules
| `(tactic| attack) => `(tactic|
  first
  | simp; done
  | ring; done
  | ring_nf; rw [Real.sq_sqrt]; ring; all_goals attack; done
  | nlinarith; done
  | apply sq_nonneg; all_goals attack; done
  | apply div_nonneg; all_goals attack; done
  | field_simp; ring; done
  | linarith; done
)

macro "obvious": tactic =>`(tactic|
  first
  | attack; done
  | congr; all_goals attack; done
  | gcongr; all_goals attack; done
  | fail "Could not prove this goal automatically. It might not be as obvious as you think!"
)

macro "in_set" : term => `(true)


macro "term_trivial": tactic =>`(tactic|
  first
  | simp; done
  | ring; done
  | ring_nf; done
  | linarith; done
  | fail "Could not prove this goal automatically. Afterall, this is an ad hoc implementation."
)

macro "old_main_hypothesis": tactic =>`(tactic|
  first
  | assumption; done
  | fail "Could not prove this goal automatically. Afterall, this is an ad hoc implementation."
)

macro "let_assigned": tactic =>`(tactic|
  first
  | dsimp; done
  | fail "Could not prove this goal automatically. Afterall, this is an ad hoc implementation."
)

macro "term_equivalence": tactic =>`(tactic|
  first
  | simp; done
  | ring; done
  | ring_nf; done
  | linarith; done
  | fail "Could not prove this goal automatically. Afterall, this is an ad hoc implementation."
)

macro "comm_ring": tactic =>`(tactic|
  first
  | ring; done
  | ring_nf; done
  | fail "Could not prove this goal automatically. Afterall, this is an ad hoc implementation."
)

macro "litnum_reduce": tactic =>`(tactic|
  first
  | simp; done
  | simp [*]; done
  | fail "Could not prove this goal automatically. Afterall, this is an ad hoc implementation."
)

macro "litnum_bound": tactic =>`(tactic|
  first
  | linarith; done
  | fail "Could not prove this goal automatically. Afterall, this is an ad hoc implementation."
)

def h (x : ℝ) : (x ^ (2:ℕ) : ℝ) ≥ ((0:ℕ) : ℝ) := by
  have h1 : (x ^ (2:ℕ) : ℝ) ≥ ((0:ℕ) : ℝ) := by
    simp
    apply sq_nonneg
  obvious

\end{lstlisting}
\end{tcolorbox}


\begin{example}
Problem:
\begin{tcolorbox}[colback=yellow!10, width=\linewidth]
Let $a\in\mathbb{R}$. Let $b\in\mathbb{R}$. Prove: $a+b=b+a$.
\end{tcolorbox}

Simplified proof:
\begin{tcolorbox}[colback=blue!10, width=\linewidth]
Trivial by the commutativity of addition for real numbers.
\end{tcolorbox}
\end{example}

Elaborated proof:
\begin{tcolorbox}[colback=green!10, width=\linewidth]
Trivial by the commutativity of addition for real numbers.
\end{tcolorbox}

Regularized proof:
\begin{tcolorbox}[colback=red!10, width=\linewidth]
Let $a\in\mathbb{R}$.
Let $b\in\mathbb{R}$.
The goal is to prove $a+b=b+a$.
We have $a+b=b+a$ by the commutativity of addition for real numbers.
\end{tcolorbox}

Lean 4 code:
\begin{tcolorbox}[colback=white!10, width=\linewidth]
\begin{lstlisting}[language=Lean4]
import Mathlib
syntax "attack" : tactic

macro_rules
| `(tactic| attack) => `(tactic|
  first
  | simp; done
  | ring; done
  | ring_nf; rw [Real.sq_sqrt]; ring; all_goals attack; done
  | nlinarith; done
  | apply sq_nonneg; all_goals attack; done
  | apply div_nonneg; all_goals attack; done
  | field_simp; ring; done
  | linarith; done
)

macro "obvious": tactic =>`(tactic|
  first
  | attack; done
  | congr; all_goals attack; done
  | gcongr; all_goals attack; done
  | fail "Could not prove this goal automatically. It might not be as obvious as you think!"
)

macro "in_set" : term => `(true)


macro "term_trivial": tactic =>`(tactic|
  first
  | simp; done
  | ring; done
  | ring_nf; done
  | linarith; done
  | fail "Could not prove this goal automatically. Afterall, this is an ad hoc implementation."
)

macro "old_main_hypothesis": tactic =>`(tactic|
  first
  | assumption; done
  | fail "Could not prove this goal automatically. Afterall, this is an ad hoc implementation."
)

macro "let_assigned": tactic =>`(tactic|
  first
  | dsimp; done
  | fail "Could not prove this goal automatically. Afterall, this is an ad hoc implementation."
)

macro "term_equivalence": tactic =>`(tactic|
  first
  | simp; done
  | ring; done
  | ring_nf; done
  | linarith; done
  | fail "Could not prove this goal automatically. Afterall, this is an ad hoc implementation."
)

macro "comm_ring": tactic =>`(tactic|
  first
  | ring; done
  | ring_nf; done
  | fail "Could not prove this goal automatically. Afterall, this is an ad hoc implementation."
)

macro "litnum_reduce": tactic =>`(tactic|
  first
  | simp; done
  | simp [*]; done
  | fail "Could not prove this goal automatically. Afterall, this is an ad hoc implementation."
)

macro "litnum_bound": tactic =>`(tactic|
  first
  | linarith; done
  | fail "Could not prove this goal automatically. Afterall, this is an ad hoc implementation."
)

def h (a b : ℝ) : a + b = b + a := by
  have h1 : a + b = b + a := by term_trivial
  obvious

\end{lstlisting}
\end{tcolorbox}


\begin{example}
Problem:
\begin{tcolorbox}[colback=yellow!10, width=\linewidth]
Let $x\in\mathbb{R}$. Assume $x> 0$. Prove: $x + \frac{1}{x} \ge 2$.
\end{tcolorbox}

Simplified proof:
\begin{tcolorbox}[colback=blue!10, width=\linewidth]
Since $x>0$, we have $(x-1)^2 \ge 0$. Thus, $x - 2 + \frac{1}{x} \ge 0$, so $x + \frac{1}{x} \ge 2$.
\end{tcolorbox}
\end{example}

Elaborated proof:
\begin{tcolorbox}[colback=green!10, width=\linewidth]
Since $x>0$, we have $(x-1)^2 = x^2 - 2x + 1 \ge 0$. Thus, $x^2 - 2x + 1 \ge 0$, so $x^2 + 1 \ge 2x$. Since $x > 0$, dividing both sides by $x$ gives $x + \frac{1}{x} \ge 2$.
\end{tcolorbox}

Regularized proof:
\begin{tcolorbox}[colback=red!10, width=\linewidth]
Let $x\in\mathbb{R}$.
Assume $x>0$.
The goal is to prove $x + \frac{1}{x} \ge 2$.
We have $x>0$.
We have ${{(x-1)}}^2 = x^2 - 2x + 1 \ge 0$.
We have $x^2 - 2x + 1 \ge 0$.
We have $x^2 + 1 \ge 2x$.
We have $x > 0$.
We have $x + \frac{1}{x} \ge 2$.
\end{tcolorbox}

Lean 4 code:
\begin{tcolorbox}[colback=white!10, width=\linewidth]
\begin{lstlisting}[language=Lean4]
import Mathlib
syntax "attack" : tactic

macro_rules
| `(tactic| attack) => `(tactic|
  first
  | simp; done
  | ring; done
  | ring_nf; rw [Real.sq_sqrt]; ring; all_goals attack; done
  | nlinarith; done
  | apply sq_nonneg; all_goals attack; done
  | apply div_nonneg; all_goals attack; done
  | field_simp; ring; done
  | linarith; done
)

macro "obvious": tactic =>`(tactic|
  first
  | attack; done
  | congr; all_goals attack; done
  | gcongr; all_goals attack; done
  | fail "Could not prove this goal automatically. It might not be as obvious as you think!"
)

macro "in_set" : term => `(true)


macro "term_trivial": tactic =>`(tactic|
  first
  | simp; done
  | ring; done
  | ring_nf; done
  | linarith; done
  | fail "Could not prove this goal automatically. Afterall, this is an ad hoc implementation."
)

macro "old_main_hypothesis": tactic =>`(tactic|
  first
  | assumption; done
  | fail "Could not prove this goal automatically. Afterall, this is an ad hoc implementation."
)

macro "let_assigned": tactic =>`(tactic|
  first
  | dsimp; done
  | fail "Could not prove this goal automatically. Afterall, this is an ad hoc implementation."
)

macro "term_equivalence": tactic =>`(tactic|
  first
  | simp; done
  | ring; done
  | ring_nf; done
  | linarith; done
  | fail "Could not prove this goal automatically. Afterall, this is an ad hoc implementation."
)

macro "comm_ring": tactic =>`(tactic|
  first
  | ring; done
  | ring_nf; done
  | fail "Could not prove this goal automatically. Afterall, this is an ad hoc implementation."
)

macro "litnum_reduce": tactic =>`(tactic|
  first
  | simp; done
  | simp [*]; done
  | fail "Could not prove this goal automatically. Afterall, this is an ad hoc implementation."
)

macro "litnum_bound": tactic =>`(tactic|
  first
  | linarith; done
  | fail "Could not prove this goal automatically. Afterall, this is an ad hoc implementation."
)

def h (x : ℝ) (h1 : x > ((0:ℕ) : ℝ)) : x + (((1:ℕ) : ℝ) / x : ℝ) ≥ ((2:ℕ) : ℝ) := by
  have h1 : x > ((0:ℕ) : ℝ) := by old_main_hypothesis
  first
  | have h2 : ((x - ((1:ℕ) : ℝ) : ℝ) ^ (2:ℕ) : ℝ) ≥ ((0:ℕ) : ℝ) := by calc
    (x - ((1:ℕ) : ℝ) : ℝ) ^ (2:ℕ) : ℝ = ((x ^ (2:ℕ) : ℝ) - ((2:ℕ) : ℝ) * x : ℝ) + ((1:ℕ) : ℝ) := by obvious
    _ ≥ (0:ℕ) : ℝ := by obvious
  | have h3 : ((x ^ (2:ℕ) : ℝ) - ((2:ℕ) : ℝ) * x : ℝ) + ((1:ℕ) : ℝ) ≥ ((0:ℕ) : ℝ) := by calc
    ((x ^ (2:ℕ) : ℝ) - ((2:ℕ) : ℝ) * x : ℝ) + ((1:ℕ) : ℝ) = (x - ((1:ℕ) : ℝ) : ℝ) ^ (2:ℕ) : ℝ := by obvious
    _ ≥ (0:ℕ) : ℝ := by obvious
  have h4 : ((x ^ (2:ℕ) : ℝ) - ((2:ℕ) : ℝ) * x : ℝ) + ((1:ℕ) : ℝ) ≥ ((0:ℕ) : ℝ) := by obvious
  have h5 : (x ^ (2:ℕ) : ℝ) + ((1:ℕ) : ℝ) ≥ ((2:ℕ) : ℝ) * x := by
    have d : x = x := by term_derivation_reflection
    have d1 : (2:ℕ) = (2:ℕ) := by term_derivation_reflection
    have d2 : (x ^ (2:ℕ) : ℝ) = ((1:ℕ) : ℝ) * (x ^ (2:ℕ) : ℝ) := by term_derivation_non_reduced_power
    have d3 : (2:ℕ) = (2:ℕ) := by term_derivation_reflection
    have d4 : x = x := by term_derivation_reflection
    have d5 : ((2:ℕ) : ℝ) * x = ((2:ℕ) : ℝ) * (x ^ (1:ℕ) : ℝ) := by term_derivation_non_one_literal_mul_atom
    have d6 : ((2:ℕ) : ℝ) * x = ((2:ℕ) : ℝ) * (x ^ (1:ℕ) : ℝ) := by term_derivation_mul_eq d3 d4 d5 eq_nat_to_real_coercion eq_identity_coercion
    have d7 : (-(((2:ℕ) : ℝ) * (x ^ (1:ℕ) : ℝ)) : ℝ) = ((-2:ℤ) : ℝ) * (x ^ (1:ℕ) : ℝ) := by term_derivation_neg_product
    have d8 : (-(((2:ℕ) : ℝ) * x) : ℝ) = ((-2:ℤ) : ℝ) * (x ^ (1:ℕ) : ℝ) := by term_derivation_neg_eq
    have d9 : ((1:ℕ) : ℝ) * (x ^ (2:ℕ) : ℝ) + ((-2:ℤ) : ℝ) * (x ^ (1:ℕ) : ℝ) = ((0:ℕ) : ℝ) + ((-2:ℤ) : ℝ) * (x ^ (1:ℕ) : ℝ) + ((1:ℕ) : ℝ) * (x ^ (2:ℕ) : ℝ) := by term_derivation_product_add_product_greater
    have d10 : (x ^ (2:ℕ) : ℝ) + (-(((2:ℕ) : ℝ) * x) : ℝ) = ((0:ℕ) : ℝ) + ((-2:ℤ) : ℝ) * (x ^ (1:ℕ) : ℝ) + ((1:ℕ) : ℝ) * (x ^ (2:ℕ) : ℝ) := by term_derivation_add_eq d2 d8 eq_identity_coercion eq_identity_coercion d9
    have d11 : ((x ^ (2:ℕ) : ℝ) - ((2:ℕ) : ℝ) * x : ℝ) = ((0:ℕ) : ℝ) + ((-2:ℤ) : ℝ) * (x ^ (1:ℕ) : ℝ) + ((1:ℕ) : ℝ) * (x ^ (2:ℕ) : ℝ) := by term_derivation_sub_eqs_add_neg d10 neg_identity_coercion
    have d12 : (1:ℕ) = (1:ℕ) := by term_derivation_reflection
    have d13 : (1:ℕ) = (1:ℕ) := by term_derivation_reflection
    have d14 : (0:ℕ) + (1:ℕ) = (1:ℕ) := by term_derivation_zero_add
    have d15 : ((1:ℕ) : ℝ) + ((-2:ℤ) : ℝ) * (x ^ (1:ℕ) : ℝ) = ((1:ℕ) : ℝ) + ((-2:ℤ) : ℝ) * (x ^ (1:ℕ) : ℝ) := by term_derivation_reflection
    have d16 : ((0:ℕ) : ℝ) + ((-2:ℤ) : ℝ) * (x ^ (1:ℕ) : ℝ) + ((1:ℕ) : ℝ) = ((1:ℕ) : ℝ) + ((-2:ℤ) : ℝ) * (x ^ (1:ℕ) : ℝ) := by term_derivation_sum_add_literal
    have d17 : ((1:ℕ) : ℝ) + ((-2:ℤ) : ℝ) * (x ^ (1:ℕ) : ℝ) + ((1:ℕ) : ℝ) * (x ^ (2:ℕ) : ℝ) = ((1:ℕ) : ℝ) + ((-2:ℤ) : ℝ) * (x ^ (1:ℕ) : ℝ) + ((1:ℕ) : ℝ) * (x ^ (2:ℕ) : ℝ) := by term_derivation_reflection
    have d18 : ((0:ℕ) : ℝ) + ((-2:ℤ) : ℝ) * (x ^ (1:ℕ) : ℝ) + ((1:ℕ) : ℝ) * (x ^ (2:ℕ) : ℝ) + ((1:ℕ) : ℝ) = ((1:ℕ) : ℝ) + ((-2:ℤ) : ℝ) * (x ^ (1:ℕ) : ℝ) + ((1:ℕ) : ℝ) * (x ^ (2:ℕ) : ℝ) := by term_derivation_sum_add_literal
    have d19 : ((x ^ (2:ℕ) : ℝ) - ((2:ℕ) : ℝ) * x : ℝ) + ((1:ℕ) : ℝ) = ((1:ℕ) : ℝ) + ((-2:ℤ) : ℝ) * (x ^ (1:ℕ) : ℝ) + ((1:ℕ) : ℝ) * (x ^ (2:ℕ) : ℝ) := by term_derivation_add_eq d11 d12 eq_identity_coercion eq_nat_to_real_coercion d18
    have d20 : (0:ℕ) = (0:ℕ) := by term_derivation_reflection
    have d21 : (1:ℕ) = (1:ℕ) := by term_derivation_reflection
    have d22 : (-2:ℤ) = (-2:ℤ) := by term_derivation_reflection
    have d23 : x = x := by term_derivation_reflection
    have d24 : (x ^ (1:ℕ) : ℝ) = x := by term_derivation_power_one
    have d25 : ((-2:ℤ) : ℝ) * x = ((-2:ℤ) : ℝ) * (x ^ (1:ℕ) : ℝ) := by term_derivation_non_one_literal_mul_atom
    have d26 : ((-2:ℤ) : ℝ) * (x ^ (1:ℕ) : ℝ) = ((-2:ℤ) : ℝ) * (x ^ (1:ℕ) : ℝ) := by term_derivation_mul_eq d22 d24 d25 eq_int_to_real_coercion eq_identity_coercion
    have d27 : ((1:ℕ) : ℝ) + ((-2:ℤ) : ℝ) * (x ^ (1:ℕ) : ℝ) = ((1:ℕ) : ℝ) + ((-2:ℤ) : ℝ) * (x ^ (1:ℕ) : ℝ) := by term_derivation_reflection
    have d28 : ((1:ℕ) : ℝ) + ((-2:ℤ) : ℝ) * (x ^ (1:ℕ) : ℝ) = ((1:ℕ) : ℝ) + ((-2:ℤ) : ℝ) * (x ^ (1:ℕ) : ℝ) := by term_derivation_add_eq d21 d26 eq_nat_to_real_coercion eq_identity_coercion d27
    have d29 : x = x := by term_derivation_reflection
    have d30 : (2:ℕ) = (2:ℕ) := by term_derivation_reflection
    have d31 : (x ^ (2:ℕ) : ℝ) = ((1:ℕ) : ℝ) * (x ^ (2:ℕ) : ℝ) := by term_derivation_non_reduced_power
    have d32 : ((1:ℕ) : ℝ) * (x ^ (2:ℕ) : ℝ) = ((1:ℕ) : ℝ) * (x ^ (2:ℕ) : ℝ) := by term_derivation_one_mul
    have d33 : ((1:ℕ) : ℝ) + ((-2:ℤ) : ℝ) * (x ^ (1:ℕ) : ℝ) + ((1:ℕ) : ℝ) * (x ^ (2:ℕ) : ℝ) = ((1:ℕ) : ℝ) + ((-2:ℤ) : ℝ) * (x ^ (1:ℕ) : ℝ) + ((1:ℕ) : ℝ) * (x ^ (2:ℕ) : ℝ) := by term_derivation_reflection
    have d34 : ((1:ℕ) : ℝ) + ((-2:ℤ) : ℝ) * (x ^ (1:ℕ) : ℝ) + ((1:ℕ) : ℝ) * (x ^ (2:ℕ) : ℝ) = ((1:ℕ) : ℝ) + ((-2:ℤ) : ℝ) * (x ^ (1:ℕ) : ℝ) + ((1:ℕ) : ℝ) * (x ^ (2:ℕ) : ℝ) := by term_derivation_add_eq d28 d32 eq_identity_coercion eq_identity_coercion d33
    have d35 : (-((0:ℕ) : ℤ) : ℤ) = (0:ℕ) := by term_derivation_neg_literal
    have d36 : ((1:ℕ) : ℝ) + ((-2:ℤ) : ℝ) * (x ^ (1:ℕ) : ℝ) + ((1:ℕ) : ℝ) * (x ^ (2:ℕ) : ℝ) + ((0:ℕ) : ℝ) = ((1:ℕ) : ℝ) + ((-2:ℤ) : ℝ) * (x ^ (1:ℕ) : ℝ) + ((1:ℕ) : ℝ) * (x ^ (2:ℕ) : ℝ) := by term_derivation_nf_add_zero
    have d37 : ((1:ℕ) : ℝ) + ((-2:ℤ) : ℝ) * (x ^ (1:ℕ) : ℝ) + ((1:ℕ) : ℝ) * (x ^ (2:ℕ) : ℝ) + ((-((0:ℕ) : ℤ) : ℤ) : ℝ) = ((1:ℕ) : ℝ) + ((-2:ℤ) : ℝ) * (x ^ (1:ℕ) : ℝ) + ((1:ℕ) : ℝ) * (x ^ (2:ℕ) : ℝ) := by term_derivation_add_eq d34 d35 eq_identity_coercion eq_int_to_real_coercion d36
    have d38 : (((1:ℕ) : ℝ) + ((-2:ℤ) : ℝ) * (x ^ (1:ℕ) : ℝ) + ((1:ℕ) : ℝ) * (x ^ (2:ℕ) : ℝ) - ((0:ℕ) : ℝ) : ℝ) = ((1:ℕ) : ℝ) + ((-2:ℤ) : ℝ) * (x ^ (1:ℕ) : ℝ) + ((1:ℕ) : ℝ) * (x ^ (2:ℕ) : ℝ) := by term_derivation_sub_eqs_add_neg d37 neg_nat_to_real_coercion
    have d39 : ((x ^ (2:ℕ) : ℝ) - ((2:ℕ) : ℝ) * x : ℝ) + ((1:ℕ) : ℝ) ≥ ((0:ℕ) : ℝ) ↔ ((1:ℕ) : ℝ) + ((-2:ℤ) : ℝ) * (x ^ (1:ℕ) : ℝ) + ((1:ℕ) : ℝ) * (x ^ (2:ℕ) : ℝ) ≥ ((0:ℕ) : ℝ) := by term_derivation_num_comparison
    have d40 : x = x := by term_derivation_reflection
    have d41 : (2:ℕ) = (2:ℕ) := by term_derivation_reflection
    have d42 : (x ^ (2:ℕ) : ℝ) = ((1:ℕ) : ℝ) * (x ^ (2:ℕ) : ℝ) := by term_derivation_non_reduced_power
    have d43 : (1:ℕ) = (1:ℕ) := by term_derivation_reflection
    have d44 : ((1:ℕ) : ℝ) * (x ^ (2:ℕ) : ℝ) + ((1:ℕ) : ℝ) = ((1:ℕ) : ℝ) + ((1:ℕ) : ℝ) * (x ^ (2:ℕ) : ℝ) := by term_derivation_product_add_literal
    have d45 : (x ^ (2:ℕ) : ℝ) + ((1:ℕ) : ℝ) = ((1:ℕ) : ℝ) + ((1:ℕ) : ℝ) * (x ^ (2:ℕ) : ℝ) := by term_derivation_add_eq d42 d43 eq_identity_coercion eq_nat_to_real_coercion d44
    have d46 : (2:ℕ) = (2:ℕ) := by term_derivation_reflection
    have d47 : x = x := by term_derivation_reflection
    have d48 : ((2:ℕ) : ℝ) * x = ((2:ℕ) : ℝ) * (x ^ (1:ℕ) : ℝ) := by term_derivation_non_one_literal_mul_atom
    have d49 : ((2:ℕ) : ℝ) * x = ((2:ℕ) : ℝ) * (x ^ (1:ℕ) : ℝ) := by term_derivation_mul_eq d46 d47 d48 eq_nat_to_real_coercion eq_identity_coercion
    have d50 : (1:ℕ) = (1:ℕ) := by term_derivation_reflection
    have d51 : x = x := by term_derivation_reflection
    have d52 : (2:ℕ) = (2:ℕ) := by term_derivation_reflection
    have d53 : (x ^ (2:ℕ) : ℝ) = ((1:ℕ) : ℝ) * (x ^ (2:ℕ) : ℝ) := by term_derivation_non_reduced_power
    have d54 : ((1:ℕ) : ℝ) * (x ^ (2:ℕ) : ℝ) = ((1:ℕ) : ℝ) * (x ^ (2:ℕ) : ℝ) := by term_derivation_one_mul
    have d55 : ((1:ℕ) : ℝ) + ((1:ℕ) : ℝ) * (x ^ (2:ℕ) : ℝ) = ((1:ℕ) : ℝ) + ((1:ℕ) : ℝ) * (x ^ (2:ℕ) : ℝ) := by term_derivation_reflection
    have d56 : ((1:ℕ) : ℝ) + ((1:ℕ) : ℝ) * (x ^ (2:ℕ) : ℝ) = ((1:ℕ) : ℝ) + ((1:ℕ) : ℝ) * (x ^ (2:ℕ) : ℝ) := by term_derivation_add_eq d50 d54 eq_nat_to_real_coercion eq_identity_coercion d55
    have d57 : (2:ℕ) = (2:ℕ) := by term_derivation_reflection
    have d58 : x = x := by term_derivation_reflection
    have d59 : (x ^ (1:ℕ) : ℝ) = x := by term_derivation_power_one
    have d60 : ((2:ℕ) : ℝ) * x = ((2:ℕ) : ℝ) * (x ^ (1:ℕ) : ℝ) := by term_derivation_non_one_literal_mul_atom
    have d61 : ((2:ℕ) : ℝ) * (x ^ (1:ℕ) : ℝ) = ((2:ℕ) : ℝ) * (x ^ (1:ℕ) : ℝ) := by term_derivation_mul_eq d57 d59 d60 eq_nat_to_real_coercion eq_identity_coercion
    have d62 : (-(((2:ℕ) : ℝ) * (x ^ (1:ℕ) : ℝ)) : ℝ) = ((-2:ℤ) : ℝ) * (x ^ (1:ℕ) : ℝ) := by term_derivation_neg_product
    have d63 : (-(((2:ℕ) : ℝ) * (x ^ (1:ℕ) : ℝ)) : ℝ) = ((-2:ℤ) : ℝ) * (x ^ (1:ℕ) : ℝ) := by term_derivation_neg_eq
    have d64 : ((1:ℕ) : ℝ) + ((-2:ℤ) : ℝ) * (x ^ (1:ℕ) : ℝ) = ((1:ℕ) : ℝ) + ((-2:ℤ) : ℝ) * (x ^ (1:ℕ) : ℝ) := by term_derivation_reflection
    have d65 : ((1:ℕ) : ℝ) + ((-2:ℤ) : ℝ) * (x ^ (1:ℕ) : ℝ) + ((1:ℕ) : ℝ) * (x ^ (2:ℕ) : ℝ) = ((1:ℕ) : ℝ) + ((-2:ℤ) : ℝ) * (x ^ (1:ℕ) : ℝ) + ((1:ℕ) : ℝ) * (x ^ (2:ℕ) : ℝ) := by term_derivation_reflection
    have d66 : ((1:ℕ) : ℝ) + ((1:ℕ) : ℝ) * (x ^ (2:ℕ) : ℝ) + ((-2:ℤ) : ℝ) * (x ^ (1:ℕ) : ℝ) = ((1:ℕ) : ℝ) + ((-2:ℤ) : ℝ) * (x ^ (1:ℕ) : ℝ) + ((1:ℕ) : ℝ) * (x ^ (2:ℕ) : ℝ) := by term_derivation_sum_add_product_greater
    have d67 : ((1:ℕ) : ℝ) + ((1:ℕ) : ℝ) * (x ^ (2:ℕ) : ℝ) + (-(((2:ℕ) : ℝ) * (x ^ (1:ℕ) : ℝ)) : ℝ) = ((1:ℕ) : ℝ) + ((-2:ℤ) : ℝ) * (x ^ (1:ℕ) : ℝ) + ((1:ℕ) : ℝ) * (x ^ (2:ℕ) : ℝ) := by term_derivation_add_eq d56 d63 eq_identity_coercion eq_identity_coercion d66
    have d68 : (((1:ℕ) : ℝ) + ((1:ℕ) : ℝ) * (x ^ (2:ℕ) : ℝ) - ((2:ℕ) : ℝ) * (x ^ (1:ℕ) : ℝ) : ℝ) = ((1:ℕ) : ℝ) + ((-2:ℤ) : ℝ) * (x ^ (1:ℕ) : ℝ) + ((1:ℕ) : ℝ) * (x ^ (2:ℕ) : ℝ) := by term_derivation_sub_eqs_add_neg d67 neg_identity_coercion
    have d69 : (x ^ (2:ℕ) : ℝ) + ((1:ℕ) : ℝ) ≥ ((2:ℕ) : ℝ) * x ↔ ((1:ℕ) : ℝ) + ((-2:ℤ) : ℝ) * (x ^ (1:ℕ) : ℝ) + ((1:ℕ) : ℝ) * (x ^ (2:ℕ) : ℝ) ≥ ((0:ℕ) : ℝ) := by term_derivation_num_comparison
    have d70 : (x ^ (2:ℕ) : ℝ) + ((1:ℕ) : ℝ) ≥ ((2:ℕ) : ℝ) * x := by term_derivation_non_trivial_hypothesis_equivalence h4 d39 d69
    assumption
  have h1 : x > ((0:ℕ) : ℝ) := by old_main_hypothesis
  have h6 : x + (((1:ℕ) : ℝ) / x : ℝ) ≥ ((2:ℕ) : ℝ) := by obvious
  obvious

\end{lstlisting}
\end{tcolorbox}


\begin{example}
Problem:
\begin{tcolorbox}[colback=yellow!10, width=\linewidth]
Let $x\in\mathbb{R}$. Prove: $x^2 + 1\ge 2x$.
\end{tcolorbox}

Simplified proof:
\begin{tcolorbox}[colback=blue!10, width=\linewidth]
$(x-1)^2 \ge 0$, so $x^2 + 1 \ge 2x$.
\end{tcolorbox}
\end{example}

Elaborated proof:
\begin{tcolorbox}[colback=green!10, width=\linewidth]
$(x-1)^2 = x^2 - 2x + 1 \ge 0$, so $x^2 + 1 \ge 2x$.
\end{tcolorbox}

Regularized proof:
\begin{tcolorbox}[colback=red!10, width=\linewidth]
Let $x\in\mathbb{R}$.
The goal is to prove $x^2 + 1 \ge 2x$.
We have ${{x-1}}^2 = x^2 - 2x + 1 \ge 0$.
We have $x^2 + 1 \ge 2x$.
\end{tcolorbox}

Lean 4 code:
\begin{tcolorbox}[colback=white!10, width=\linewidth]
\begin{lstlisting}[language=Lean4]
import Mathlib
syntax "attack" : tactic

macro_rules
| `(tactic| attack) => `(tactic|
  first
  | simp; done
  | ring; done
  | ring_nf; rw [Real.sq_sqrt]; ring; all_goals attack; done
  | nlinarith; done
  | apply sq_nonneg; all_goals attack; done
  | apply div_nonneg; all_goals attack; done
  | field_simp; ring; done
  | linarith; done
)

macro "obvious": tactic =>`(tactic|
  first
  | attack; done
  | congr; all_goals attack; done
  | gcongr; all_goals attack; done
  | fail "Could not prove this goal automatically. It might not be as obvious as you think!"
)

macro "in_set" : term => `(true)


macro "term_trivial": tactic =>`(tactic|
  first
  | simp; done
  | ring; done
  | ring_nf; done
  | linarith; done
  | fail "Could not prove this goal automatically. Afterall, this is an ad hoc implementation."
)

macro "old_main_hypothesis": tactic =>`(tactic|
  first
  | assumption; done
  | fail "Could not prove this goal automatically. Afterall, this is an ad hoc implementation."
)

macro "let_assigned": tactic =>`(tactic|
  first
  | dsimp; done
  | fail "Could not prove this goal automatically. Afterall, this is an ad hoc implementation."
)

macro "term_equivalence": tactic =>`(tactic|
  first
  | simp; done
  | ring; done
  | ring_nf; done
  | linarith; done
  | fail "Could not prove this goal automatically. Afterall, this is an ad hoc implementation."
)

macro "comm_ring": tactic =>`(tactic|
  first
  | ring; done
  | ring_nf; done
  | fail "Could not prove this goal automatically. Afterall, this is an ad hoc implementation."
)

macro "litnum_reduce": tactic =>`(tactic|
  first
  | simp; done
  | simp [*]; done
  | fail "Could not prove this goal automatically. Afterall, this is an ad hoc implementation."
)

macro "litnum_bound": tactic =>`(tactic|
  first
  | linarith; done
  | fail "Could not prove this goal automatically. Afterall, this is an ad hoc implementation."
)

def h (x : ℝ) : (x ^ (2:ℕ) : ℝ) + ((1:ℕ) : ℝ) ≥ ((2:ℕ) : ℝ) * x := by
  first
  | have h1 : ((x - ((1:ℕ) : ℝ) : ℝ) ^ (2:ℕ) : ℝ) ≥ ((0:ℕ) : ℝ) := by calc
    (x - ((1:ℕ) : ℝ) : ℝ) ^ (2:ℕ) : ℝ = ((x ^ (2:ℕ) : ℝ) - ((2:ℕ) : ℝ) * x : ℝ) + ((1:ℕ) : ℝ) := by obvious
    _ ≥ (0:ℕ) : ℝ := by obvious
  | have h2 : ((x ^ (2:ℕ) : ℝ) - ((2:ℕ) : ℝ) * x : ℝ) + ((1:ℕ) : ℝ) ≥ ((0:ℕ) : ℝ) := by calc
    ((x ^ (2:ℕ) : ℝ) - ((2:ℕ) : ℝ) * x : ℝ) + ((1:ℕ) : ℝ) = (x - ((1:ℕ) : ℝ) : ℝ) ^ (2:ℕ) : ℝ := by obvious
    _ ≥ (0:ℕ) : ℝ := by obvious
  have h3 : (x ^ (2:ℕ) : ℝ) + ((1:ℕ) : ℝ) ≥ ((2:ℕ) : ℝ) * x := by obvious
  obvious

\end{lstlisting}
\end{tcolorbox}


\begin{example}
Problem:
\begin{tcolorbox}[colback=yellow!10, width=\linewidth]
Let $x\in\mathbb{R}$. Prove: $x + 1 \ge 2\sqrt{x}$.
\end{tcolorbox}

Simplified proof:
\begin{tcolorbox}[colback=blue!10, width=\linewidth]
We have $(\sqrt{x}-1)^2 \ge 0$, so $x+1 \ge 2\sqrt{x}$.
\end{tcolorbox}
\end{example}

Elaborated proof:
\begin{tcolorbox}[colback=green!10, width=\linewidth]
We have $(\sqrt{x}-1)^2 = x - 2\sqrt{x} + 1 \ge 0$, so $x+1 \ge 2\sqrt{x}$.
\end{tcolorbox}

Regularized proof:
\begin{tcolorbox}[colback=red!10, width=\linewidth]
Let $x\in\mathbb{R}$.
Assume $x\ge 0$.
The goal is to prove $x+1 \ge 2\sqrt{x}$.
We have ${(\sqrt{x}-1)}^2 = x - 2\sqrt{x} + 1 \ge 0$.
We have $x+1 \ge 2\sqrt{x}$.
\end{tcolorbox}

Lean 4 code:
\begin{tcolorbox}[colback=white!10, width=\linewidth]
\begin{lstlisting}[language=Lean4]
import Mathlib
syntax "attack" : tactic

macro_rules
| `(tactic| attack) => `(tactic|
  first
  | simp; done
  | ring; done
  | ring_nf; rw [Real.sq_sqrt]; ring; all_goals attack; done
  | nlinarith; done
  | apply sq_nonneg; all_goals attack; done
  | apply div_nonneg; all_goals attack; done
  | field_simp; ring; done
  | linarith; done
)

macro "obvious": tactic =>`(tactic|
  first
  | attack; done
  | congr; all_goals attack; done
  | gcongr; all_goals attack; done
  | fail "Could not prove this goal automatically. It might not be as obvious as you think!"
)

macro "in_set" : term => `(true)


macro "term_trivial": tactic =>`(tactic|
  first
  | simp; done
  | ring; done
  | ring_nf; done
  | linarith; done
  | fail "Could not prove this goal automatically. Afterall, this is an ad hoc implementation."
)

macro "old_main_hypothesis": tactic =>`(tactic|
  first
  | assumption; done
  | fail "Could not prove this goal automatically. Afterall, this is an ad hoc implementation."
)

macro "let_assigned": tactic =>`(tactic|
  first
  | dsimp; done
  | fail "Could not prove this goal automatically. Afterall, this is an ad hoc implementation."
)

macro "term_equivalence": tactic =>`(tactic|
  first
  | simp; done
  | ring; done
  | ring_nf; done
  | linarith; done
  | fail "Could not prove this goal automatically. Afterall, this is an ad hoc implementation."
)

macro "comm_ring": tactic =>`(tactic|
  first
  | ring; done
  | ring_nf; done
  | fail "Could not prove this goal automatically. Afterall, this is an ad hoc implementation."
)

macro "litnum_reduce": tactic =>`(tactic|
  first
  | simp; done
  | simp [*]; done
  | fail "Could not prove this goal automatically. Afterall, this is an ad hoc implementation."
)

macro "litnum_bound": tactic =>`(tactic|
  first
  | linarith; done
  | fail "Could not prove this goal automatically. Afterall, this is an ad hoc implementation."
)

def h (x : ℝ) (h1 : x ≥ ((0:ℕ) : ℝ)) : x + ((1:ℕ) : ℝ) ≥ ((2:ℕ) : ℝ) * √ x := by
  first
  | have h2 : ((√ x - ((1:ℕ) : ℝ) : ℝ) ^ (2:ℕ) : ℝ) ≥ ((0:ℕ) : ℝ) := by calc
    (√ x - ((1:ℕ) : ℝ) : ℝ) ^ (2:ℕ) : ℝ = (x - ((2:ℕ) : ℝ) * √ x : ℝ) + ((1:ℕ) : ℝ) := by obvious
    _ ≥ (0:ℕ) : ℝ := by obvious
  | have h3 : (x - ((2:ℕ) : ℝ) * √ x : ℝ) + ((1:ℕ) : ℝ) ≥ ((0:ℕ) : ℝ) := by calc
    (x - ((2:ℕ) : ℝ) * √ x : ℝ) + ((1:ℕ) : ℝ) = (√ x - ((1:ℕ) : ℝ) : ℝ) ^ (2:ℕ) : ℝ := by obvious
    _ ≥ (0:ℕ) : ℝ := by obvious
  have h4 : x + ((1:ℕ) : ℝ) ≥ ((2:ℕ) : ℝ) * √ x := by obvious
  obvious

\end{lstlisting}
\end{tcolorbox}


\begin{example}
Problem:
\begin{tcolorbox}[colback=yellow!10, width=\linewidth]
Let $x\in\mathbb{R}$. Assume $x>0$.
    Let $y\in\mathbb{R}$. Assume $y>0$.
    Prove: $\frac{1}{x} + \frac{1}{y} \ge \frac{4}{x+y}$.
\end{tcolorbox}

Simplified proof:
\begin{tcolorbox}[colback=blue!10, width=\linewidth]
Since $x>0$ and $y>0$, we have $(x-y)^2 \ge 0$. Thus $(x+y)^2 \ge 4xy$. Dividing by $xy(x+y)$ gives $\frac{1}{x}+\frac{1}{y} \ge \frac{4}{x+y}$.
\end{tcolorbox}
\end{example}

Elaborated proof:
\begin{tcolorbox}[colback=green!10, width=\linewidth]
Since $x>0$ and $y>0$, we have $(x-y)^2 \ge 0$. Thus $(x-y)^2 = x^2 -2xy + y^2 \ge 0$, so $x^2 + y^2 \ge 2xy$, and hence $x^2 + 2xy + y^2 = (x+y)^2 \ge 4xy$. Dividing by $xy(x+y)$ gives $\frac{(x+y)^2}{xy(x+y)} = \frac{x+y}{xy} = \frac{x}{xy} + \frac{y}{xy} = \frac{1}{y}+\frac{1}{x} \ge \frac{4xy}{xy(x+y)} = \frac{4}{x+y}$.
\end{tcolorbox}

Regularized proof:
\begin{tcolorbox}[colback=red!10, width=\linewidth]
Let $x\in\mathbb{R}$.
Assume $x>0$.
Let $y\in\mathbb{R}$.
Assume $y>0$.
The goal is to prove $\frac{1}{x} + \frac{1}{y} \ge \frac{4}{x+y}$.
We have $x \in \mathbb{R}$.
We have $x>0$.
We have $y \in \mathbb{R}$.
We have $y>0$.
It's enough to show that $\frac{1}{x} + \frac{1}{y} \ge \frac{4}{x+y}$.
We have ${(x-y)}^2 \ge 0$.
We have ${(x-y)}^2 = x^2 -2xy + y^2 \ge 0$.
We have $x^2 + y^2 \ge 2xy$.
We have $x^2 + 2xy + y^2 = {(x+y)}^2 \ge 4xy$.
We have $\frac{{(x+y)}^2}{xy(x+y)} = \frac{x+y}{xy} = \frac{x}{xy} + \frac{y}{xy} = \frac{1}{y}+\frac{1}{x} \ge \frac{4xy}{xy(x+y)} = \frac{4}{x+y}$.
\end{tcolorbox}

Lean 4 code:
\begin{tcolorbox}[colback=white!10, width=\linewidth]
\begin{lstlisting}[language=Lean4]
import Mathlib
syntax "attack" : tactic

macro_rules
| `(tactic| attack) => `(tactic|
  first
  | simp; done
  | ring; done
  | ring_nf; rw [Real.sq_sqrt]; ring; all_goals attack; done
  | nlinarith; done
  | apply sq_nonneg; all_goals attack; done
  | apply div_nonneg; all_goals attack; done
  | field_simp; ring; done
  | linarith; done
)

macro "obvious": tactic =>`(tactic|
  first
  | attack; done
  | congr; all_goals attack; done
  | gcongr; all_goals attack; done
  | fail "Could not prove this goal automatically. It might not be as obvious as you think!"
)

macro "in_set" : term => `(true)


macro "term_trivial": tactic =>`(tactic|
  first
  | simp; done
  | ring; done
  | ring_nf; done
  | linarith; done
  | fail "Could not prove this goal automatically. Afterall, this is an ad hoc implementation."
)

macro "old_main_hypothesis": tactic =>`(tactic|
  first
  | assumption; done
  | fail "Could not prove this goal automatically. Afterall, this is an ad hoc implementation."
)

macro "let_assigned": tactic =>`(tactic|
  first
  | dsimp; done
  | fail "Could not prove this goal automatically. Afterall, this is an ad hoc implementation."
)

macro "term_equivalence": tactic =>`(tactic|
  first
  | simp; done
  | ring; done
  | ring_nf; done
  | linarith; done
  | fail "Could not prove this goal automatically. Afterall, this is an ad hoc implementation."
)

macro "comm_ring": tactic =>`(tactic|
  first
  | ring; done
  | ring_nf; done
  | fail "Could not prove this goal automatically. Afterall, this is an ad hoc implementation."
)

macro "litnum_reduce": tactic =>`(tactic|
  first
  | simp; done
  | simp [*]; done
  | fail "Could not prove this goal automatically. Afterall, this is an ad hoc implementation."
)

macro "litnum_bound": tactic =>`(tactic|
  first
  | linarith; done
  | fail "Could not prove this goal automatically. Afterall, this is an ad hoc implementation."
)

def h (x : ℝ) (h1 : x > ((0:ℕ) : ℝ)) (y : ℝ) (h2 : y > ((0:ℕ) : ℝ)) : (((1:ℕ) : ℝ) / x : ℝ) + (((1:ℕ) : ℝ) / y : ℝ) ≥ (((4:ℕ) : ℝ) / (x + y) : ℝ) := by
  have h3 : in_set := by obvious
  have h1 : x > ((0:ℕ) : ℝ) := by old_main_hypothesis
  have h4 : in_set := by obvious
  have h2 : y > ((0:ℕ) : ℝ) := by old_main_hypothesis
  have h13 : (((1:ℕ) : ℝ) / x : ℝ) + (((1:ℕ) : ℝ) / y : ℝ) ≥ (((4:ℕ) : ℝ) / (x + y) : ℝ) := by
    have h5 : ((x - y : ℝ) ^ (2:ℕ) : ℝ) ≥ ((0:ℕ) : ℝ) := by
      simp
      apply sq_nonneg
    first
    | have h6 : ((x - y : ℝ) ^ (2:ℕ) : ℝ) ≥ ((0:ℕ) : ℝ) := by calc
      (x - y : ℝ) ^ (2:ℕ) : ℝ = ((x ^ (2:ℕ) : ℝ) - ((2:ℕ) : ℝ) * x * y : ℝ) + (y ^ (2:ℕ) : ℝ) := by obvious
      _ ≥ (0:ℕ) : ℝ := by obvious
    | have h7 : ((x ^ (2:ℕ) : ℝ) - ((2:ℕ) : ℝ) * x * y : ℝ) + (y ^ (2:ℕ) : ℝ) ≥ ((0:ℕ) : ℝ) := by calc
      ((x ^ (2:ℕ) : ℝ) - ((2:ℕ) : ℝ) * x * y : ℝ) + (y ^ (2:ℕ) : ℝ) = (x - y : ℝ) ^ (2:ℕ) : ℝ := by obvious
      _ ≥ (0:ℕ) : ℝ := by obvious
    have h8 : (x ^ (2:ℕ) : ℝ) + (y ^ (2:ℕ) : ℝ) ≥ ((2:ℕ) : ℝ) * x * y := by obvious
    first
    | have h9 : (x ^ (2:ℕ) : ℝ) + ((2:ℕ) : ℝ) * x * y + (y ^ (2:ℕ) : ℝ) ≥ ((4:ℕ) : ℝ) * x * y := by calc
      (x ^ (2:ℕ) : ℝ) + ((2:ℕ) : ℝ) * x * y + (y ^ (2:ℕ) : ℝ) = (x + y) ^ (2:ℕ) : ℝ := by obvious
      _ ≥ ((4:ℕ) : ℝ) * x * y := by obvious
    | have h10 : ((x + y) ^ (2:ℕ) : ℝ) ≥ ((4:ℕ) : ℝ) * x * y := by calc
      (x + y) ^ (2:ℕ) : ℝ = (x ^ (2:ℕ) : ℝ) + ((2:ℕ) : ℝ) * x * y + (y ^ (2:ℕ) : ℝ) := by obvious
      _ ≥ ((4:ℕ) : ℝ) * x * y := by obvious
    first
    | have h11 : (((x + y) ^ (2:ℕ) : ℝ) / (x * y * (x + y)) : ℝ) ≥ (((4:ℕ) : ℝ) / (x + y) : ℝ) := by calc
      ((x + y) ^ (2:ℕ) : ℝ) / (x * y * (x + y)) : ℝ = (x + y) / (x * y) : ℝ := by obvious
      _ = (x / (x * y) : ℝ) + (y / (x * y) : ℝ) := by obvious
      _ = (((1:ℕ) : ℝ) / y : ℝ) + (((1:ℕ) : ℝ) / x : ℝ) := by obvious
      _ ≥ ((4:ℕ) : ℝ) * x * y / (x * y * (x + y)) : ℝ := by obvious
      _ = ((4:ℕ) : ℝ) / (x + y) : ℝ := by obvious
    | have h12 : (((1:ℕ) : ℝ) / y : ℝ) + (((1:ℕ) : ℝ) / x : ℝ) ≥ (((4:ℕ) : ℝ) / (x + y) : ℝ) := by calc
      (((1:ℕ) : ℝ) / y : ℝ) + (((1:ℕ) : ℝ) / x : ℝ) = (x / (x * y) : ℝ) + (y / (x * y) : ℝ) := by obvious
      _ = (x + y) / (x * y) : ℝ := by obvious
      _ = ((x + y) ^ (2:ℕ) : ℝ) / (x * y * (x + y)) : ℝ := by obvious
      _ ≥ ((4:ℕ) : ℝ) * x * y / (x * y * (x + y)) : ℝ := by obvious
      _ = ((4:ℕ) : ℝ) / (x + y) : ℝ := by obvious
    obvious
  have h14 : (((1:ℕ) : ℝ) / x : ℝ) + (((1:ℕ) : ℝ) / y : ℝ) ≥ (((4:ℕ) : ℝ) / (x + y) : ℝ) := by obvious

\end{lstlisting}
\end{tcolorbox}


\begin{example}
Problem:
\begin{tcolorbox}[colback=yellow!10, width=\linewidth]
Let $a\in\mathbb{R}$. Assume $a > 0$.
    Let $b\in\mathbb{R}$. Assume $b > 0$.
    Prove: $\frac{a}{b} + \frac{b}{a} \ge 2$.
\end{tcolorbox}

Simplified proof:
\begin{tcolorbox}[colback=blue!10, width=\linewidth]
Since $a>0$ and $b>0$, we have $(\sqrt{a/b} - \sqrt{b/a})^2 \ge 0$. Thus $a/b + b/a \ge 2$.
\end{tcolorbox}
\end{example}

Elaborated proof:
\begin{tcolorbox}[colback=green!10, width=\linewidth]
Since $a>0$ and $b>0$, we have $(\sqrt{a/b} - \sqrt{b/a})^2 = (\sqrt{a/b})^2 - 2\sqrt{a/b}\sqrt{b/a} + (\sqrt{b/a})^2 = a/b - 2 + b/a \ge 0$. Thus $a/b - 2 + b/a \ge 0$, so $a/b + b/a \ge 2$.
\end{tcolorbox}

Regularized proof:
\begin{tcolorbox}[colback=red!10, width=\linewidth]
Let $a\in\mathbb{R}$.
Let $b\in\mathbb{R}$.
Assume $a>0$.
Assume $b>0$.
The goal is to prove $a/b + b/a \ge 2$.
We have $a>0$.
We have $b>0$.
We have ${{(\sqrt{a/b} - \sqrt{b/a})}}^2 = {{(\sqrt{a/b})}}^2 - 2\sqrt{a/b}\sqrt{b/a} + {{(\sqrt{b/a})}}^2 = a/b - 2 + b/a \ge 0$.
We have $a/b - 2 + b/a \ge 0$.
We have $a/b + b/a \ge 2$.
\end{tcolorbox}

Lean 4 code:
\begin{tcolorbox}[colback=white!10, width=\linewidth]
\begin{lstlisting}[language=Lean4]
import Mathlib
syntax "attack" : tactic

macro_rules
| `(tactic| attack) => `(tactic|
  first
  | simp; done
  | ring; done
  | ring_nf; rw [Real.sq_sqrt]; ring; all_goals attack; done
  | nlinarith; done
  | apply sq_nonneg; all_goals attack; done
  | apply div_nonneg; all_goals attack; done
  | field_simp; ring; done
  | linarith; done
)

macro "obvious": tactic =>`(tactic|
  first
  | attack; done
  | congr; all_goals attack; done
  | gcongr; all_goals attack; done
  | fail "Could not prove this goal automatically. It might not be as obvious as you think!"
)

macro "in_set" : term => `(true)


macro "term_trivial": tactic =>`(tactic|
  first
  | simp; done
  | ring; done
  | ring_nf; done
  | linarith; done
  | fail "Could not prove this goal automatically. Afterall, this is an ad hoc implementation."
)

macro "old_main_hypothesis": tactic =>`(tactic|
  first
  | assumption; done
  | fail "Could not prove this goal automatically. Afterall, this is an ad hoc implementation."
)

macro "let_assigned": tactic =>`(tactic|
  first
  | dsimp; done
  | fail "Could not prove this goal automatically. Afterall, this is an ad hoc implementation."
)

macro "term_equivalence": tactic =>`(tactic|
  first
  | simp; done
  | ring; done
  | ring_nf; done
  | linarith; done
  | fail "Could not prove this goal automatically. Afterall, this is an ad hoc implementation."
)

macro "comm_ring": tactic =>`(tactic|
  first
  | ring; done
  | ring_nf; done
  | fail "Could not prove this goal automatically. Afterall, this is an ad hoc implementation."
)

macro "litnum_reduce": tactic =>`(tactic|
  first
  | simp; done
  | simp [*]; done
  | fail "Could not prove this goal automatically. Afterall, this is an ad hoc implementation."
)

macro "litnum_bound": tactic =>`(tactic|
  first
  | linarith; done
  | fail "Could not prove this goal automatically. Afterall, this is an ad hoc implementation."
)

def h (a b : ℝ) (h1 : a > ((0:ℕ) : ℝ)) (h2 : b > ((0:ℕ) : ℝ)) : (a / b : ℝ) + (b / a : ℝ) ≥ ((2:ℕ) : ℝ) := by
  have h1 : a > ((0:ℕ) : ℝ) := by old_main_hypothesis
  have h2 : b > ((0:ℕ) : ℝ) := by old_main_hypothesis
  first
  | have h3 : ((√ (a / b : ℝ) - √ (b / a : ℝ) : ℝ) ^ (2:ℕ) : ℝ) ≥ ((0:ℕ) : ℝ) := by calc
    (√ (a / b : ℝ) - √ (b / a : ℝ) : ℝ) ^ (2:ℕ) : ℝ = ((√ (a / b : ℝ) ^ (2:ℕ) : ℝ) - ((2:ℕ) : ℝ) * √ (a / b : ℝ) * √ (b / a : ℝ) : ℝ) + (√ (b / a : ℝ) ^ (2:ℕ) : ℝ) := by obvious
    _ = ((a / b : ℝ) - ((2:ℕ) : ℝ) : ℝ) + (b / a : ℝ) := by obvious
    _ ≥ (0:ℕ) : ℝ := by obvious
  | have h4 : ((a / b : ℝ) - ((2:ℕ) : ℝ) : ℝ) + (b / a : ℝ) ≥ ((0:ℕ) : ℝ) := by calc
    ((a / b : ℝ) - ((2:ℕ) : ℝ) : ℝ) + (b / a : ℝ) = ((√ (a / b : ℝ) ^ (2:ℕ) : ℝ) - ((2:ℕ) : ℝ) * √ (a / b : ℝ) * √ (b / a : ℝ) : ℝ) + (√ (b / a : ℝ) ^ (2:ℕ) : ℝ) := by obvious
    _ = (√ (a / b : ℝ) - √ (b / a : ℝ) : ℝ) ^ (2:ℕ) : ℝ := by obvious
    _ ≥ (0:ℕ) : ℝ := by obvious
  have h5 : ((a / b : ℝ) - ((2:ℕ) : ℝ) : ℝ) + (b / a : ℝ) ≥ ((0:ℕ) : ℝ) := by obvious
  have h6 : (a / b : ℝ) + (b / a : ℝ) ≥ ((2:ℕ) : ℝ) := by
    have d : a = a := by term_derivation_reflection
    have d1 : b = b := by term_derivation_reflection
    have d2 : a * (b ^ (-1:ℤ) : ℝ) = ((1:ℕ) : ℝ) * ((b ^ (-1:ℤ) : ℝ) * (a ^ (1:ℕ) : ℝ)) := by term_derivation_atom_mul_exponential_greater
    have d3 : (a / b : ℝ) = ((1:ℕ) : ℝ) * ((b ^ (-1:ℤ) : ℝ) * (a ^ (1:ℕ) : ℝ)) := by term_derivation_div_atom
    have d4 : (a / b : ℝ) = ((1:ℕ) : ℝ) * ((b ^ (-1:ℤ) : ℝ) * (a ^ (1:ℕ) : ℝ)) := by term_derivation_div_eq d d1 eq_identity_coercion eq_identity_coercion d3
    have d5 : (-((2:ℕ) : ℤ) : ℤ) = (-2:ℤ) := by term_derivation_neg_literal
    have d6 : ((1:ℕ) : ℝ) * ((b ^ (-1:ℤ) : ℝ) * (a ^ (1:ℕ) : ℝ)) + ((-2:ℤ) : ℝ) = ((-2:ℤ) : ℝ) + ((1:ℕ) : ℝ) * ((b ^ (-1:ℤ) : ℝ) * (a ^ (1:ℕ) : ℝ)) := by term_derivation_product_add_literal
    have d7 : (a / b : ℝ) + ((-((2:ℕ) : ℤ) : ℤ) : ℝ) = ((-2:ℤ) : ℝ) + ((1:ℕ) : ℝ) * ((b ^ (-1:ℤ) : ℝ) * (a ^ (1:ℕ) : ℝ)) := by term_derivation_add_eq d4 d5 eq_identity_coercion eq_int_to_real_coercion d6
    have d8 : ((a / b : ℝ) - ((2:ℕ) : ℝ) : ℝ) = ((-2:ℤ) : ℝ) + ((1:ℕ) : ℝ) * ((b ^ (-1:ℤ) : ℝ) * (a ^ (1:ℕ) : ℝ)) := by term_derivation_sub_eqs_add_neg d7 neg_nat_to_real_coercion
    have d9 : b = b := by term_derivation_reflection
    have d10 : a = a := by term_derivation_reflection
    have d11 : b * (a ^ (-1:ℤ) : ℝ) = ((1:ℕ) : ℝ) * ((b ^ (1:ℕ) : ℝ) * (a ^ (-1:ℤ) : ℝ)) := by term_derivation_atom_mul_exponential_less
    have d12 : (b / a : ℝ) = ((1:ℕ) : ℝ) * ((b ^ (1:ℕ) : ℝ) * (a ^ (-1:ℤ) : ℝ)) := by term_derivation_div_atom
    have d13 : (b / a : ℝ) = ((1:ℕ) : ℝ) * ((b ^ (1:ℕ) : ℝ) * (a ^ (-1:ℤ) : ℝ)) := by term_derivation_div_eq d9 d10 eq_identity_coercion eq_identity_coercion d12
    have d14 : ((-2:ℤ) : ℝ) + ((1:ℕ) : ℝ) * ((b ^ (-1:ℤ) : ℝ) * (a ^ (1:ℕ) : ℝ)) + ((1:ℕ) : ℝ) * ((b ^ (1:ℕ) : ℝ) * (a ^ (-1:ℤ) : ℝ)) = ((-2:ℤ) : ℝ) + ((1:ℕ) : ℝ) * ((b ^ (-1:ℤ) : ℝ) * (a ^ (1:ℕ) : ℝ)) + ((1:ℕ) : ℝ) * ((b ^ (1:ℕ) : ℝ) * (a ^ (-1:ℤ) : ℝ)) := by term_derivation_reflection
    have d15 : ((a / b : ℝ) - ((2:ℕ) : ℝ) : ℝ) + (b / a : ℝ) = ((-2:ℤ) : ℝ) + ((1:ℕ) : ℝ) * ((b ^ (-1:ℤ) : ℝ) * (a ^ (1:ℕ) : ℝ)) + ((1:ℕ) : ℝ) * ((b ^ (1:ℕ) : ℝ) * (a ^ (-1:ℤ) : ℝ)) := by term_derivation_add_eq d8 d13 eq_identity_coercion eq_identity_coercion d14
    have d16 : (0:ℕ) = (0:ℕ) := by term_derivation_reflection
    have d17 : (-2:ℤ) = (-2:ℤ) := by term_derivation_reflection
    have d18 : b = b := by term_derivation_reflection
    have d19 : (-1:ℤ) = (-1:ℤ) := by term_derivation_reflection
    have d20 : (b ^ (-1:ℤ) : ℝ) = ((1:ℕ) : ℝ) * (b ^ (-1:ℤ) : ℝ) := by term_derivation_non_reduced_power
    have d21 : a = a := by term_derivation_reflection
    have d22 : (a ^ (1:ℕ) : ℝ) = a := by term_derivation_power_one
    have d23 : ((1:ℕ) : ℝ) * (b ^ (-1:ℤ) : ℝ) * a = ((1:ℕ) : ℝ) * ((b ^ (-1:ℤ) : ℝ) * (a ^ (1:ℕ) : ℝ)) := by term_derivation_simple_product_mul_base_less
    have d24 : (b ^ (-1:ℤ) : ℝ) * (a ^ (1:ℕ) : ℝ) = ((1:ℕ) : ℝ) * ((b ^ (-1:ℤ) : ℝ) * (a ^ (1:ℕ) : ℝ)) := by term_derivation_mul_eq d20 d22 d23 eq_identity_coercion eq_identity_coercion
    have d25 : ((1:ℕ) : ℝ) * ((b ^ (-1:ℤ) : ℝ) * (a ^ (1:ℕ) : ℝ)) = ((1:ℕ) : ℝ) * ((b ^ (-1:ℤ) : ℝ) * (a ^ (1:ℕ) : ℝ)) := by term_derivation_one_mul
    have d26 : ((-2:ℤ) : ℝ) + ((1:ℕ) : ℝ) * ((b ^ (-1:ℤ) : ℝ) * (a ^ (1:ℕ) : ℝ)) = ((-2:ℤ) : ℝ) + ((1:ℕ) : ℝ) * ((b ^ (-1:ℤ) : ℝ) * (a ^ (1:ℕ) : ℝ)) := by term_derivation_reflection
    have d27 : ((-2:ℤ) : ℝ) + ((1:ℕ) : ℝ) * ((b ^ (-1:ℤ) : ℝ) * (a ^ (1:ℕ) : ℝ)) = ((-2:ℤ) : ℝ) + ((1:ℕ) : ℝ) * ((b ^ (-1:ℤ) : ℝ) * (a ^ (1:ℕ) : ℝ)) := by term_derivation_add_eq d17 d25 eq_int_to_real_coercion eq_identity_coercion d26
    have d28 : b = b := by term_derivation_reflection
    have d29 : (b ^ (1:ℕ) : ℝ) = b := by term_derivation_power_one
    have d30 : a = a := by term_derivation_reflection
    have d31 : (-1:ℤ) = (-1:ℤ) := by term_derivation_reflection
    have d32 : (a ^ (-1:ℤ) : ℝ) = ((1:ℕ) : ℝ) * (a ^ (-1:ℤ) : ℝ) := by term_derivation_non_reduced_power
    have d33 : b * ((1:ℕ) : ℝ) = b := by term_derivation_mul_one
    have d34 : b * (a ^ (-1:ℤ) : ℝ) = ((1:ℕ) : ℝ) * ((b ^ (1:ℕ) : ℝ) * (a ^ (-1:ℤ) : ℝ)) := by term_derivation_atom_mul_exponential_less
    have d35 : b * (((1:ℕ) : ℝ) * (a ^ (-1:ℤ) : ℝ)) = ((1:ℕ) : ℝ) * ((b ^ (1:ℕ) : ℝ) * (a ^ (-1:ℤ) : ℝ)) := by term_derivation_mul_product d33 d34
    have d36 : (b ^ (1:ℕ) : ℝ) * (a ^ (-1:ℤ) : ℝ) = ((1:ℕ) : ℝ) * ((b ^ (1:ℕ) : ℝ) * (a ^ (-1:ℤ) : ℝ)) := by term_derivation_mul_eq d29 d32 d35 eq_identity_coercion eq_identity_coercion
    have d37 : ((1:ℕ) : ℝ) * ((b ^ (1:ℕ) : ℝ) * (a ^ (-1:ℤ) : ℝ)) = ((1:ℕ) : ℝ) * ((b ^ (1:ℕ) : ℝ) * (a ^ (-1:ℤ) : ℝ)) := by term_derivation_one_mul
    have d38 : ((-2:ℤ) : ℝ) + ((1:ℕ) : ℝ) * ((b ^ (-1:ℤ) : ℝ) * (a ^ (1:ℕ) : ℝ)) + ((1:ℕ) : ℝ) * ((b ^ (1:ℕ) : ℝ) * (a ^ (-1:ℤ) : ℝ)) = ((-2:ℤ) : ℝ) + ((1:ℕ) : ℝ) * ((b ^ (-1:ℤ) : ℝ) * (a ^ (1:ℕ) : ℝ)) + ((1:ℕ) : ℝ) * ((b ^ (1:ℕ) : ℝ) * (a ^ (-1:ℤ) : ℝ)) := by term_derivation_reflection
    have d39 : ((-2:ℤ) : ℝ) + ((1:ℕ) : ℝ) * ((b ^ (-1:ℤ) : ℝ) * (a ^ (1:ℕ) : ℝ)) + ((1:ℕ) : ℝ) * ((b ^ (1:ℕ) : ℝ) * (a ^ (-1:ℤ) : ℝ)) = ((-2:ℤ) : ℝ) + ((1:ℕ) : ℝ) * ((b ^ (-1:ℤ) : ℝ) * (a ^ (1:ℕ) : ℝ)) + ((1:ℕ) : ℝ) * ((b ^ (1:ℕ) : ℝ) * (a ^ (-1:ℤ) : ℝ)) := by term_derivation_add_eq d27 d37 eq_identity_coercion eq_identity_coercion d38
    have d40 : (-((0:ℕ) : ℤ) : ℤ) = (0:ℕ) := by term_derivation_neg_literal
    have d41 : ((-2:ℤ) : ℝ) + ((1:ℕ) : ℝ) * ((b ^ (-1:ℤ) : ℝ) * (a ^ (1:ℕ) : ℝ)) + ((1:ℕ) : ℝ) * ((b ^ (1:ℕ) : ℝ) * (a ^ (-1:ℤ) : ℝ)) + ((0:ℕ) : ℝ) = ((-2:ℤ) : ℝ) + ((1:ℕ) : ℝ) * ((b ^ (-1:ℤ) : ℝ) * (a ^ (1:ℕ) : ℝ)) + ((1:ℕ) : ℝ) * ((b ^ (1:ℕ) : ℝ) * (a ^ (-1:ℤ) : ℝ)) := by term_derivation_nf_add_zero
    have d42 : ((-2:ℤ) : ℝ) + ((1:ℕ) : ℝ) * ((b ^ (-1:ℤ) : ℝ) * (a ^ (1:ℕ) : ℝ)) + ((1:ℕ) : ℝ) * ((b ^ (1:ℕ) : ℝ) * (a ^ (-1:ℤ) : ℝ)) + ((-((0:ℕ) : ℤ) : ℤ) : ℝ) = ((-2:ℤ) : ℝ) + ((1:ℕ) : ℝ) * ((b ^ (-1:ℤ) : ℝ) * (a ^ (1:ℕ) : ℝ)) + ((1:ℕ) : ℝ) * ((b ^ (1:ℕ) : ℝ) * (a ^ (-1:ℤ) : ℝ)) := by term_derivation_add_eq d39 d40 eq_identity_coercion eq_int_to_real_coercion d41
    have d43 : (((-2:ℤ) : ℝ) + ((1:ℕ) : ℝ) * ((b ^ (-1:ℤ) : ℝ) * (a ^ (1:ℕ) : ℝ)) + ((1:ℕ) : ℝ) * ((b ^ (1:ℕ) : ℝ) * (a ^ (-1:ℤ) : ℝ)) - ((0:ℕ) : ℝ) : ℝ) = ((-2:ℤ) : ℝ) + ((1:ℕ) : ℝ) * ((b ^ (-1:ℤ) : ℝ) * (a ^ (1:ℕ) : ℝ)) + ((1:ℕ) : ℝ) * ((b ^ (1:ℕ) : ℝ) * (a ^ (-1:ℤ) : ℝ)) := by term_derivation_sub_eqs_add_neg d42 neg_nat_to_real_coercion
    have d44 : ((a / b : ℝ) - ((2:ℕ) : ℝ) : ℝ) + (b / a : ℝ) ≥ ((0:ℕ) : ℝ) ↔ ((-2:ℤ) : ℝ) + ((1:ℕ) : ℝ) * ((b ^ (-1:ℤ) : ℝ) * (a ^ (1:ℕ) : ℝ)) + ((1:ℕ) : ℝ) * ((b ^ (1:ℕ) : ℝ) * (a ^ (-1:ℤ) : ℝ)) ≥ ((0:ℕ) : ℝ) := by term_derivation_num_comparison
    have d45 : a = a := by term_derivation_reflection
    have d46 : b = b := by term_derivation_reflection
    have d47 : a * (b ^ (-1:ℤ) : ℝ) = ((1:ℕ) : ℝ) * ((b ^ (-1:ℤ) : ℝ) * (a ^ (1:ℕ) : ℝ)) := by term_derivation_atom_mul_exponential_greater
    have d48 : (a / b : ℝ) = ((1:ℕ) : ℝ) * ((b ^ (-1:ℤ) : ℝ) * (a ^ (1:ℕ) : ℝ)) := by term_derivation_div_atom
    have d49 : (a / b : ℝ) = ((1:ℕ) : ℝ) * ((b ^ (-1:ℤ) : ℝ) * (a ^ (1:ℕ) : ℝ)) := by term_derivation_div_eq d45 d46 eq_identity_coercion eq_identity_coercion d48
    have d50 : b = b := by term_derivation_reflection
    have d51 : a = a := by term_derivation_reflection
    have d52 : b * (a ^ (-1:ℤ) : ℝ) = ((1:ℕ) : ℝ) * ((b ^ (1:ℕ) : ℝ) * (a ^ (-1:ℤ) : ℝ)) := by term_derivation_atom_mul_exponential_less
    have d53 : (b / a : ℝ) = ((1:ℕ) : ℝ) * ((b ^ (1:ℕ) : ℝ) * (a ^ (-1:ℤ) : ℝ)) := by term_derivation_div_atom
    have d54 : (b / a : ℝ) = ((1:ℕ) : ℝ) * ((b ^ (1:ℕ) : ℝ) * (a ^ (-1:ℤ) : ℝ)) := by term_derivation_div_eq d50 d51 eq_identity_coercion eq_identity_coercion d53
    have d55 : ((1:ℕ) : ℝ) * ((b ^ (-1:ℤ) : ℝ) * (a ^ (1:ℕ) : ℝ)) + ((1:ℕ) : ℝ) * ((b ^ (1:ℕ) : ℝ) * (a ^ (-1:ℤ) : ℝ)) = ((0:ℕ) : ℝ) + ((1:ℕ) : ℝ) * ((b ^ (-1:ℤ) : ℝ) * (a ^ (1:ℕ) : ℝ)) + ((1:ℕ) : ℝ) * ((b ^ (1:ℕ) : ℝ) * (a ^ (-1:ℤ) : ℝ)) := by term_derivation_product_add_product_less
    have d56 : (a / b : ℝ) + (b / a : ℝ) = ((0:ℕ) : ℝ) + ((1:ℕ) : ℝ) * ((b ^ (-1:ℤ) : ℝ) * (a ^ (1:ℕ) : ℝ)) + ((1:ℕ) : ℝ) * ((b ^ (1:ℕ) : ℝ) * (a ^ (-1:ℤ) : ℝ)) := by term_derivation_add_eq d49 d54 eq_identity_coercion eq_identity_coercion d55
    have d57 : (2:ℕ) = (2:ℕ) := by term_derivation_reflection
    have d58 : b = b := by term_derivation_reflection
    have d59 : (-1:ℤ) = (-1:ℤ) := by term_derivation_reflection
    have d60 : (b ^ (-1:ℤ) : ℝ) = ((1:ℕ) : ℝ) * (b ^ (-1:ℤ) : ℝ) := by term_derivation_non_reduced_power
    have d61 : a = a := by term_derivation_reflection
    have d62 : (a ^ (1:ℕ) : ℝ) = a := by term_derivation_power_one
    have d63 : ((1:ℕ) : ℝ) * (b ^ (-1:ℤ) : ℝ) * a = ((1:ℕ) : ℝ) * ((b ^ (-1:ℤ) : ℝ) * (a ^ (1:ℕ) : ℝ)) := by term_derivation_simple_product_mul_base_less
    have d64 : (b ^ (-1:ℤ) : ℝ) * (a ^ (1:ℕ) : ℝ) = ((1:ℕ) : ℝ) * ((b ^ (-1:ℤ) : ℝ) * (a ^ (1:ℕ) : ℝ)) := by term_derivation_mul_eq d60 d62 d63 eq_identity_coercion eq_identity_coercion
    have d65 : ((1:ℕ) : ℝ) * ((b ^ (-1:ℤ) : ℝ) * (a ^ (1:ℕ) : ℝ)) = ((1:ℕ) : ℝ) * ((b ^ (-1:ℤ) : ℝ) * (a ^ (1:ℕ) : ℝ)) := by term_derivation_one_mul
    have d66 : ((0:ℕ) : ℝ) + ((1:ℕ) : ℝ) * ((b ^ (-1:ℤ) : ℝ) * (a ^ (1:ℕ) : ℝ)) = ((1:ℕ) : ℝ) * ((b ^ (-1:ℤ) : ℝ) * (a ^ (1:ℕ) : ℝ)) := by term_derivation_zero_add
    have d67 : b = b := by term_derivation_reflection
    have d68 : (b ^ (1:ℕ) : ℝ) = b := by term_derivation_power_one
    have d69 : a = a := by term_derivation_reflection
    have d70 : (-1:ℤ) = (-1:ℤ) := by term_derivation_reflection
    have d71 : (a ^ (-1:ℤ) : ℝ) = ((1:ℕ) : ℝ) * (a ^ (-1:ℤ) : ℝ) := by term_derivation_non_reduced_power
    have d72 : b * ((1:ℕ) : ℝ) = b := by term_derivation_mul_one
    have d73 : b * (a ^ (-1:ℤ) : ℝ) = ((1:ℕ) : ℝ) * ((b ^ (1:ℕ) : ℝ) * (a ^ (-1:ℤ) : ℝ)) := by term_derivation_atom_mul_exponential_less
    have d74 : b * (((1:ℕ) : ℝ) * (a ^ (-1:ℤ) : ℝ)) = ((1:ℕ) : ℝ) * ((b ^ (1:ℕ) : ℝ) * (a ^ (-1:ℤ) : ℝ)) := by term_derivation_mul_product d72 d73
    have d75 : (b ^ (1:ℕ) : ℝ) * (a ^ (-1:ℤ) : ℝ) = ((1:ℕ) : ℝ) * ((b ^ (1:ℕ) : ℝ) * (a ^ (-1:ℤ) : ℝ)) := by term_derivation_mul_eq d68 d71 d74 eq_identity_coercion eq_identity_coercion
    have d76 : ((1:ℕ) : ℝ) * ((b ^ (1:ℕ) : ℝ) * (a ^ (-1:ℤ) : ℝ)) = ((1:ℕ) : ℝ) * ((b ^ (1:ℕ) : ℝ) * (a ^ (-1:ℤ) : ℝ)) := by term_derivation_one_mul
    have d77 : ((1:ℕ) : ℝ) * ((b ^ (-1:ℤ) : ℝ) * (a ^ (1:ℕ) : ℝ)) + ((1:ℕ) : ℝ) * ((b ^ (1:ℕ) : ℝ) * (a ^ (-1:ℤ) : ℝ)) = ((0:ℕ) : ℝ) + ((1:ℕ) : ℝ) * ((b ^ (-1:ℤ) : ℝ) * (a ^ (1:ℕ) : ℝ)) + ((1:ℕ) : ℝ) * ((b ^ (1:ℕ) : ℝ) * (a ^ (-1:ℤ) : ℝ)) := by term_derivation_product_add_product_less
    have d78 : ((0:ℕ) : ℝ) + ((1:ℕ) : ℝ) * ((b ^ (-1:ℤ) : ℝ) * (a ^ (1:ℕ) : ℝ)) + ((1:ℕ) : ℝ) * ((b ^ (1:ℕ) : ℝ) * (a ^ (-1:ℤ) : ℝ)) = ((0:ℕ) : ℝ) + ((1:ℕ) : ℝ) * ((b ^ (-1:ℤ) : ℝ) * (a ^ (1:ℕ) : ℝ)) + ((1:ℕ) : ℝ) * ((b ^ (1:ℕ) : ℝ) * (a ^ (-1:ℤ) : ℝ)) := by term_derivation_add_eq d66 d76 eq_identity_coercion eq_identity_coercion d77
    have d79 : (-((2:ℕ) : ℤ) : ℤ) = (-2:ℤ) := by term_derivation_neg_literal
    have d80 : (-2:ℤ) = (-2:ℤ) := by term_derivation_reflection
    have d81 : ((0:ℕ) : ℤ) + (-2:ℤ) = (-2:ℤ) := by term_derivation_zero_add
    have d82 : ((-2:ℤ) : ℝ) + ((1:ℕ) : ℝ) * ((b ^ (-1:ℤ) : ℝ) * (a ^ (1:ℕ) : ℝ)) = ((-2:ℤ) : ℝ) + ((1:ℕ) : ℝ) * ((b ^ (-1:ℤ) : ℝ) * (a ^ (1:ℕ) : ℝ)) := by term_derivation_reflection
    have d83 : ((0:ℕ) : ℝ) + ((1:ℕ) : ℝ) * ((b ^ (-1:ℤ) : ℝ) * (a ^ (1:ℕ) : ℝ)) + ((-2:ℤ) : ℝ) = ((-2:ℤ) : ℝ) + ((1:ℕ) : ℝ) * ((b ^ (-1:ℤ) : ℝ) * (a ^ (1:ℕ) : ℝ)) := by term_derivation_sum_add_literal
    have d84 : ((-2:ℤ) : ℝ) + ((1:ℕ) : ℝ) * ((b ^ (-1:ℤ) : ℝ) * (a ^ (1:ℕ) : ℝ)) + ((1:ℕ) : ℝ) * ((b ^ (1:ℕ) : ℝ) * (a ^ (-1:ℤ) : ℝ)) = ((-2:ℤ) : ℝ) + ((1:ℕ) : ℝ) * ((b ^ (-1:ℤ) : ℝ) * (a ^ (1:ℕ) : ℝ)) + ((1:ℕ) : ℝ) * ((b ^ (1:ℕ) : ℝ) * (a ^ (-1:ℤ) : ℝ)) := by term_derivation_reflection
    have d85 : ((0:ℕ) : ℝ) + ((1:ℕ) : ℝ) * ((b ^ (-1:ℤ) : ℝ) * (a ^ (1:ℕ) : ℝ)) + ((1:ℕ) : ℝ) * ((b ^ (1:ℕ) : ℝ) * (a ^ (-1:ℤ) : ℝ)) + ((-2:ℤ) : ℝ) = ((-2:ℤ) : ℝ) + ((1:ℕ) : ℝ) * ((b ^ (-1:ℤ) : ℝ) * (a ^ (1:ℕ) : ℝ)) + ((1:ℕ) : ℝ) * ((b ^ (1:ℕ) : ℝ) * (a ^ (-1:ℤ) : ℝ)) := by term_derivation_sum_add_literal
    have d86 : ((0:ℕ) : ℝ) + ((1:ℕ) : ℝ) * ((b ^ (-1:ℤ) : ℝ) * (a ^ (1:ℕ) : ℝ)) + ((1:ℕ) : ℝ) * ((b ^ (1:ℕ) : ℝ) * (a ^ (-1:ℤ) : ℝ)) + ((-((2:ℕ) : ℤ) : ℤ) : ℝ) = ((-2:ℤ) : ℝ) + ((1:ℕ) : ℝ) * ((b ^ (-1:ℤ) : ℝ) * (a ^ (1:ℕ) : ℝ)) + ((1:ℕ) : ℝ) * ((b ^ (1:ℕ) : ℝ) * (a ^ (-1:ℤ) : ℝ)) := by term_derivation_add_eq d78 d79 eq_identity_coercion eq_int_to_real_coercion d85
    have d87 : (((0:ℕ) : ℝ) + ((1:ℕ) : ℝ) * ((b ^ (-1:ℤ) : ℝ) * (a ^ (1:ℕ) : ℝ)) + ((1:ℕ) : ℝ) * ((b ^ (1:ℕ) : ℝ) * (a ^ (-1:ℤ) : ℝ)) - ((2:ℕ) : ℝ) : ℝ) = ((-2:ℤ) : ℝ) + ((1:ℕ) : ℝ) * ((b ^ (-1:ℤ) : ℝ) * (a ^ (1:ℕ) : ℝ)) + ((1:ℕ) : ℝ) * ((b ^ (1:ℕ) : ℝ) * (a ^ (-1:ℤ) : ℝ)) := by term_derivation_sub_eqs_add_neg d86 neg_nat_to_real_coercion
    have d88 : (a / b : ℝ) + (b / a : ℝ) ≥ ((2:ℕ) : ℝ) ↔ ((-2:ℤ) : ℝ) + ((1:ℕ) : ℝ) * ((b ^ (-1:ℤ) : ℝ) * (a ^ (1:ℕ) : ℝ)) + ((1:ℕ) : ℝ) * ((b ^ (1:ℕ) : ℝ) * (a ^ (-1:ℤ) : ℝ)) ≥ ((0:ℕ) : ℝ) := by term_derivation_num_comparison
    have d89 : (a / b : ℝ) + (b / a : ℝ) ≥ ((2:ℕ) : ℝ) := by term_derivation_non_trivial_hypothesis_equivalence h5 d44 d88
    assumption
  obvious

\end{lstlisting}
\end{tcolorbox}


\begin{example}
Problem:
\begin{tcolorbox}[colback=yellow!10, width=\linewidth]
Let $x\in\mathbb{R}$. Let $y\in\mathbb{R}$.
    Prove: $x^2 + y^2 \ge 2xy$.
\end{tcolorbox}

Simplified proof:
\begin{tcolorbox}[colback=blue!10, width=\linewidth]
Trivial since $(x-y)^2 \ge 0$.
\end{tcolorbox}
\end{example}

Elaborated proof:
\begin{tcolorbox}[colback=green!10, width=\linewidth]
Trivial since $(x-y)^2 = x^2 -2xy + y^2 \ge 0$, so $x^2 + y^2 \ge 2xy$.
\end{tcolorbox}

Regularized proof:
\begin{tcolorbox}[colback=red!10, width=\linewidth]
Let $x\in\mathbb{R}$.
Let $y\in\mathbb{R}$.
The goal is to prove $x^2 + y^2 \ge 2xy$.
We have ${{(x-y)}}^2 = x^2 -2xy + y^2 \ge 0$.
We have $x^2 + y^2 \ge 2xy$.
\end{tcolorbox}

Lean 4 code:
\begin{tcolorbox}[colback=white!10, width=\linewidth]
\begin{lstlisting}[language=Lean4]
import Mathlib
syntax "attack" : tactic

macro_rules
| `(tactic| attack) => `(tactic|
  first
  | simp; done
  | ring; done
  | ring_nf; rw [Real.sq_sqrt]; ring; all_goals attack; done
  | nlinarith; done
  | apply sq_nonneg; all_goals attack; done
  | apply div_nonneg; all_goals attack; done
  | field_simp; ring; done
  | linarith; done
)

macro "obvious": tactic =>`(tactic|
  first
  | attack; done
  | congr; all_goals attack; done
  | gcongr; all_goals attack; done
  | fail "Could not prove this goal automatically. It might not be as obvious as you think!"
)

macro "in_set" : term => `(true)


macro "term_trivial": tactic =>`(tactic|
  first
  | simp; done
  | ring; done
  | ring_nf; done
  | linarith; done
  | fail "Could not prove this goal automatically. Afterall, this is an ad hoc implementation."
)

macro "old_main_hypothesis": tactic =>`(tactic|
  first
  | assumption; done
  | fail "Could not prove this goal automatically. Afterall, this is an ad hoc implementation."
)

macro "let_assigned": tactic =>`(tactic|
  first
  | dsimp; done
  | fail "Could not prove this goal automatically. Afterall, this is an ad hoc implementation."
)

macro "term_equivalence": tactic =>`(tactic|
  first
  | simp; done
  | ring; done
  | ring_nf; done
  | linarith; done
  | fail "Could not prove this goal automatically. Afterall, this is an ad hoc implementation."
)

macro "comm_ring": tactic =>`(tactic|
  first
  | ring; done
  | ring_nf; done
  | fail "Could not prove this goal automatically. Afterall, this is an ad hoc implementation."
)

macro "litnum_reduce": tactic =>`(tactic|
  first
  | simp; done
  | simp [*]; done
  | fail "Could not prove this goal automatically. Afterall, this is an ad hoc implementation."
)

macro "litnum_bound": tactic =>`(tactic|
  first
  | linarith; done
  | fail "Could not prove this goal automatically. Afterall, this is an ad hoc implementation."
)

def h (x y : ℝ) : (x ^ (2:ℕ) : ℝ) + (y ^ (2:ℕ) : ℝ) ≥ ((2:ℕ) : ℝ) * x * y := by
  first
  | have h1 : ((x - y : ℝ) ^ (2:ℕ) : ℝ) ≥ ((0:ℕ) : ℝ) := by calc
    (x - y : ℝ) ^ (2:ℕ) : ℝ = ((x ^ (2:ℕ) : ℝ) - ((2:ℕ) : ℝ) * x * y : ℝ) + (y ^ (2:ℕ) : ℝ) := by obvious
    _ ≥ (0:ℕ) : ℝ := by obvious
  | have h2 : ((x ^ (2:ℕ) : ℝ) - ((2:ℕ) : ℝ) * x * y : ℝ) + (y ^ (2:ℕ) : ℝ) ≥ ((0:ℕ) : ℝ) := by calc
    ((x ^ (2:ℕ) : ℝ) - ((2:ℕ) : ℝ) * x * y : ℝ) + (y ^ (2:ℕ) : ℝ) = (x - y : ℝ) ^ (2:ℕ) : ℝ := by obvious
    _ ≥ (0:ℕ) : ℝ := by obvious
  have h3 : (x ^ (2:ℕ) : ℝ) + (y ^ (2:ℕ) : ℝ) ≥ ((2:ℕ) : ℝ) * x * y := by obvious
  obvious

\end{lstlisting}
\end{tcolorbox}
\end{document}