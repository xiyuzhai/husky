%!TEX TS-program = xelatex
\documentclass{article}
\usepackage{amsmath}
\usepackage{amssymb}
\usepackage{fvextra}
\usepackage{tcolorbox}
\usepackage{listings}
\usepackage{amsthm}
\usepackage{fontspec}  % For Unicode support
\setmonofont{DejaVu Sans Mono}  % For monospace/code blocks
\usepackage{unicode-math}
% \setmathfont{XITS Math}  % Or another Unicode math font
\newtheorem{example}{Example}
\fvset{breaklines=true}



\begin{document}
\lstdefinelanguage{Lean4}{
    breaklines=true,
    basicstyle=\ttfamily\normalsize,
    keepspaces=true,
    morekeywords={
        def, theorem, lemma, example, have, show, calc, let, assume, by, exact,
        sorry, obvious, Type, Prop, where, with, extends, class, instance,
        structure, inductive, mutual, coinductive, variable, variables,
        universe, universes, deriving, abbrev, partial, terminating,
        namespace, import, open, export, private, protected, public,
        noncomputable, unsafe, macro, syntax, macro_rules, set_option,
        attribute, local, scoped, section, end, match, fun, if, then, else,
        return, do, for, in, while, break, continue, try, catch, throw,
        mut, pure, opaque
    },
    morekeywords=[2]{
        ℕ, ℝ, ∈, ≥, ≤, →, ∀, ∃, ⊢, ∧, ∨, ¬, ≠, ×, ⊗, ⊕, ∘, □, ◇, ∎,
        ⟨, ⟩, ⦃, ⦄, ▸, ≈, ∼, ≡, ⌊, ⌋, ⌈, ⌉
    },
    morecomment=[l]--,     % Line comments start with --
    morestring=[b]",       % Strings in double quotes
    sensitive=true,         % Case-sensitive
    keywordstyle=\color{blue}\bfseries,      % Regular keywords in blue and bold
    keywordstyle=[2]\color{purple}\bfseries, % Special symbols in purple and bold
    commentstyle=\color{green!50!black},     % Comments in dark green
    stringstyle=\color{red},                 % Strings in red
}



\begin{example}
Problem:
\begin{tcolorbox}[colback=yellow!10, width=\linewidth]
Prove: $1+1=2$.
\end{tcolorbox}

Simplified proof:
\begin{tcolorbox}[colback=blue!10, width=\linewidth]
This is trivial.
\end{tcolorbox}
\end{example}

Elaborated proof:
\begin{tcolorbox}[colback=green!10, width=\linewidth]
This is trivial.
\end{tcolorbox}

Regularized proof:
\begin{tcolorbox}[colback=red!10, width=\linewidth]
The goal is to prove $1+1=2$.
We have $1+1=2$.
\end{tcolorbox}


\begin{example}
Problem:
\begin{tcolorbox}[colback=yellow!10, width=\linewidth]
Let $x\in\mathbb{R}$. Prove: $x^2\ge 0$.
\end{tcolorbox}

Simplified proof:
\begin{tcolorbox}[colback=blue!10, width=\linewidth]
The square of a real number is non-negative.
\end{tcolorbox}
\end{example}

Elaborated proof:
\begin{tcolorbox}[colback=green!10, width=\linewidth]
The square of a real number is non-negative.
\end{tcolorbox}

Regularized proof:
\begin{tcolorbox}[colback=red!10, width=\linewidth]
Let $x\in\mathbb{R}$.
The goal is to prove ${{x}}^2 \ge 0$.
We have ${{x}}^2 \ge 0$.
\end{tcolorbox}


\begin{example}
Problem:
\begin{tcolorbox}[colback=yellow!10, width=\linewidth]
Let $a\in\mathbb{R}$. Let $b\in\mathbb{R}$. Prove: $a+b=b+a$.
\end{tcolorbox}

Simplified proof:
\begin{tcolorbox}[colback=blue!10, width=\linewidth]
Trivial by the commutativity of addition for real numbers.
\end{tcolorbox}
\end{example}

Elaborated proof:
\begin{tcolorbox}[colback=green!10, width=\linewidth]
Trivial by the commutativity of addition for real numbers.
\end{tcolorbox}

Regularized proof:
\begin{tcolorbox}[colback=red!10, width=\linewidth]
Let $a\in\mathbb{R}$.
Let $b\in\mathbb{R}$.
The goal is to prove $a+b=b+a$.
We have $a+b=b+a$ by the commutativity of addition for real numbers.
\end{tcolorbox}


\begin{example}
Problem:
\begin{tcolorbox}[colback=yellow!10, width=\linewidth]
Let $x\in\mathbb{R}$. Assume $x> 0$. Prove: $x + \frac{1}{x} \ge 2$.
\end{tcolorbox}

Simplified proof:
\begin{tcolorbox}[colback=blue!10, width=\linewidth]
Since $x>0$, we have $(x-1)^2 \ge 0$. Thus, $x - 2 + \frac{1}{x} \ge 0$, so $x + \frac{1}{x} \ge 2$.
\end{tcolorbox}
\end{example}

Elaborated proof:
\begin{tcolorbox}[colback=green!10, width=\linewidth]
Since $x>0$, we have $(x-1)^2 = x^2 - 2x + 1 \ge 0$. Thus, $x^2 - 2x + 1 \ge 0$, so $x^2 + 1 \ge 2x$. Since $x > 0$, dividing both sides by $x$ gives $x + \frac{1}{x} \ge 2$.
\end{tcolorbox}

Regularized proof:
\begin{tcolorbox}[colback=red!10, width=\linewidth]
Let $x\in\mathbb{R}$.
Assume $x>0$.
The goal is to prove $x + \frac{1}{x} \ge 2$.
We have $x>0$.
We have ${{(x-1)}}^2 = x^2 - 2x + 1 \ge 0$.
We have $x^2 - 2x + 1 \ge 0$.
We have $x^2 + 1 \ge 2x$.
We have $x > 0$.
We have $x + \frac{1}{x} \ge 2$.
\end{tcolorbox}


\begin{example}
Problem:
\begin{tcolorbox}[colback=yellow!10, width=\linewidth]
Let $x\in\mathbb{R}$. Prove: $x^2 + 1\ge 2x$.
\end{tcolorbox}

Simplified proof:
\begin{tcolorbox}[colback=blue!10, width=\linewidth]
$(x-1)^2 \ge 0$, so $x^2 + 1 \ge 2x$.
\end{tcolorbox}
\end{example}

Elaborated proof:
\begin{tcolorbox}[colback=green!10, width=\linewidth]
$(x-1)^2 = x^2 - 2x + 1 \ge 0$, so $x^2 + 1 \ge 2x$.
\end{tcolorbox}

Regularized proof:
\begin{tcolorbox}[colback=red!10, width=\linewidth]
Let $x\in\mathbb{R}$.
The goal is to prove $x^2 + 1 \ge 2x$.
We have ${{x-1}}^2 = x^2 - 2x + 1 \ge 0$.
We have $x^2 + 1 \ge 2x$.
\end{tcolorbox}


\begin{example}
Problem:
\begin{tcolorbox}[colback=yellow!10, width=\linewidth]
Let $x\in\mathbb{R}$. Prove: $x + 1 \ge 2\sqrt{x}$.
\end{tcolorbox}

Simplified proof:
\begin{tcolorbox}[colback=blue!10, width=\linewidth]
We have $(\sqrt{x}-1)^2 \ge 0$, so $x+1 \ge 2\sqrt{x}$.
\end{tcolorbox}
\end{example}

Elaborated proof:
\begin{tcolorbox}[colback=green!10, width=\linewidth]
We have $(\sqrt{x}-1)^2 = x - 2\sqrt{x} + 1 \ge 0$, so $x+1 \ge 2\sqrt{x}$.
\end{tcolorbox}

Regularized proof:
\begin{tcolorbox}[colback=red!10, width=\linewidth]
Let $x\in\mathbb{R}$.
Assume $x\ge 0$.
The goal is to prove $x+1 \ge 2\sqrt{x}$.
We have ${(\sqrt{x}-1)}^2 = x - 2\sqrt{x} + 1 \ge 0$.
We have $x+1 \ge 2\sqrt{x}$.
\end{tcolorbox}


\begin{example}
Problem:
\begin{tcolorbox}[colback=yellow!10, width=\linewidth]
Let $x\in\mathbb{R}$. Assume $x>0$.
    Let $y\in\mathbb{R}$. Assume $y>0$.
    Prove: $\frac{1}{x} + \frac{1}{y} \ge \frac{4}{x+y}$.
\end{tcolorbox}

Simplified proof:
\begin{tcolorbox}[colback=blue!10, width=\linewidth]
Since $x>0$ and $y>0$, we have $(x-y)^2 \ge 0$. Thus $(x+y)^2 \ge 4xy$. Dividing by $xy(x+y)$ gives $\frac{1}{x}+\frac{1}{y} \ge \frac{4}{x+y}$.
\end{tcolorbox}
\end{example}

Elaborated proof:
\begin{tcolorbox}[colback=green!10, width=\linewidth]
Since $x>0$ and $y>0$, we have $(x-y)^2 \ge 0$. Thus $(x-y)^2 = x^2 -2xy + y^2 \ge 0$, so $x^2 + y^2 \ge 2xy$, and hence $x^2 + 2xy + y^2 = (x+y)^2 \ge 4xy$. Dividing by $xy(x+y)$ gives $\frac{(x+y)^2}{xy(x+y)} = \frac{x+y}{xy} = \frac{x}{xy} + \frac{y}{xy} = \frac{1}{y}+\frac{1}{x} \ge \frac{4xy}{xy(x+y)} = \frac{4}{x+y}$.
\end{tcolorbox}

Regularized proof:
\begin{tcolorbox}[colback=red!10, width=\linewidth]
Let $x\in\mathbb{R}$.
Assume $x>0$.
Let $y\in\mathbb{R}$.
Assume $y>0$.
The goal is to prove $\frac{1}{x} + \frac{1}{y} \ge \frac{4}{x+y}$.
We have $x \in \mathbb{R}$.
We have $x>0$.
We have $y \in \mathbb{R}$.
We have $y>0$.
It's enough to show that $\frac{1}{x} + \frac{1}{y} \ge \frac{4}{x+y}$.
We have ${(x-y)}^2 \ge 0$.
We have ${(x-y)}^2 = x^2 -2xy + y^2 \ge 0$.
We have $x^2 + y^2 \ge 2xy$.
We have $x^2 + 2xy + y^2 = {(x+y)}^2 \ge 4xy$.
We have $\frac{{(x+y)}^2}{xy(x+y)} = \frac{x+y}{xy} = \frac{x}{xy} + \frac{y}{xy} = \frac{1}{y}+\frac{1}{x} \ge \frac{4xy}{xy(x+y)} = \frac{4}{x+y}$.
\end{tcolorbox}


\begin{example}
Problem:
\begin{tcolorbox}[colback=yellow!10, width=\linewidth]
Let $a\in\mathbb{R}$. Assume $a > 0$.
    Let $b\in\mathbb{R}$. Assume $b > 0$.
    Prove: $\frac{a}{b} + \frac{b}{a} \ge 2$.
\end{tcolorbox}

Simplified proof:
\begin{tcolorbox}[colback=blue!10, width=\linewidth]
Since $a>0$ and $b>0$, we have $(\sqrt{a/b} - \sqrt{b/a})^2 \ge 0$. Thus $a/b + b/a \ge 2$.
\end{tcolorbox}
\end{example}

Elaborated proof:
\begin{tcolorbox}[colback=green!10, width=\linewidth]
Since $a>0$ and $b>0$, we have $(\sqrt{a/b} - \sqrt{b/a})^2 = (\sqrt{a/b})^2 - 2\sqrt{a/b}\sqrt{b/a} + (\sqrt{b/a})^2 = a/b - 2 + b/a \ge 0$. Thus $a/b - 2 + b/a \ge 0$, so $a/b + b/a \ge 2$.
\end{tcolorbox}

Regularized proof:
\begin{tcolorbox}[colback=red!10, width=\linewidth]
Let $a\in\mathbb{R}$.
Let $b\in\mathbb{R}$.
Assume $a>0$.
Assume $b>0$.
The goal is to prove $a/b + b/a \ge 2$.
We have $a>0$.
We have $b>0$.
We have ${{(\sqrt{a/b} - \sqrt{b/a})}}^2 = {{(\sqrt{a/b})}}^2 - 2\sqrt{a/b}\sqrt{b/a} + {{(\sqrt{b/a})}}^2 = a/b - 2 + b/a \ge 0$.
We have $a/b - 2 + b/a \ge 0$.
We have $a/b + b/a \ge 2$.
\end{tcolorbox}


\begin{example}
Problem:
\begin{tcolorbox}[colback=yellow!10, width=\linewidth]
Let $x\in\mathbb{R}$. Let $y\in\mathbb{R}$.
    Prove: $x^2 + y^2 \ge 2xy$.
\end{tcolorbox}

Simplified proof:
\begin{tcolorbox}[colback=blue!10, width=\linewidth]
Trivial since $(x-y)^2 \ge 0$.
\end{tcolorbox}
\end{example}

Elaborated proof:
\begin{tcolorbox}[colback=green!10, width=\linewidth]
Trivial since $(x-y)^2 = x^2 -2xy + y^2 \ge 0$, so $x^2 + y^2 \ge 2xy$.
\end{tcolorbox}

Regularized proof:
\begin{tcolorbox}[colback=red!10, width=\linewidth]
Let $x\in\mathbb{R}$.
Let $y\in\mathbb{R}$.
The goal is to prove $x^2 + y^2 \ge 2xy$.
We have ${{(x-y)}}^2 = x^2 -2xy + y^2 \ge 0$.
We have $x^2 + y^2 \ge 2xy$.
\end{tcolorbox}
\end{document}