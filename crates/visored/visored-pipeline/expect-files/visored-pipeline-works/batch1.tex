%!TEX TS-program = xelatex
\documentclass{article}
\usepackage{amsmath}
\usepackage{amssymb}
\usepackage{fvextra}
\usepackage{tcolorbox}
\usepackage{listings}
\usepackage{amsthm}
\usepackage{fontspec}  % For Unicode support
\setmonofont{DejaVu Sans Mono}  % For monospace/code blocks
\usepackage{unicode-math}
% \setmathfont{XITS Math}  % Or another Unicode math font
\newtheorem{example}{Example}
\fvset{breaklines=true}



\begin{document}
\lstdefinelanguage{Lean4}{
    breaklines=true,
    basicstyle=\ttfamily\normalsize,
    keepspaces=true,
    morekeywords={
        def, theorem, lemma, example, have, show, calc, let, assume, by, exact,
        sorry, obvious, Type, Prop, where, with, extends, class, instance,
        structure, inductive, mutual, coinductive, variable, variables,
        universe, universes, deriving, abbrev, partial, terminating,
        namespace, import, open, export, private, protected, public,
        noncomputable, unsafe, macro, syntax, macro_rules, set_option,
        attribute, local, scoped, section, end, match, fun, if, then, else,
        return, do, for, in, while, break, continue, try, catch, throw,
        mut, pure, opaque
    },
    morekeywords=[2]{
        ℕ, ℝ, ∈, ≥, ≤, →, ∀, ∃, ⊢, ∧, ∨, ¬, ≠, ×, ⊗, ⊕, ∘, □, ◇, ∎,
        ⟨, ⟩, ⦃, ⦄, ▸, ≈, ∼, ≡, ⌊, ⌋, ⌈, ⌉
    },
    morecomment=[l]--,     % Line comments start with --
    morestring=[b]",       % Strings in double quotes
    sensitive=true,         % Case-sensitive
    keywordstyle=\color{blue}\bfseries,      % Regular keywords in blue and bold
    keywordstyle=[2]\color{purple}\bfseries, % Special symbols in purple and bold
    commentstyle=\color{green!50!black},     % Comments in dark green
    stringstyle=\color{red},                 % Strings in red
}



\begin{example}
Problem:
\begin{tcolorbox}[colback=yellow!10, width=\linewidth]
Prove: $1+1=2$.
\end{tcolorbox}

Simplified proof:
\begin{tcolorbox}[colback=blue!10, width=\linewidth]
This is trivial.
\end{tcolorbox}
\end{example}

Elaborated proof:
\begin{tcolorbox}[colback=green!10, width=\linewidth]
This is trivial.
\end{tcolorbox}

Regularized proof:
\begin{tcolorbox}[colback=red!10, width=\linewidth]
The goal is to prove $1+1=2$.
We have $1+1=2$.
\end{tcolorbox}

Lean 4 code:
\begin{tcolorbox}[colback=white!10, width=\linewidth]
\begin{lstlisting}[language=Lean4]
import Mathlib
syntax "attack" : tactic

macro_rules
| `(tactic| attack) => `(tactic|
  first
  | simp; done
  | ring; done
  | ring_nf; rw [Real.sq_sqrt]; ring; all_goals attack; done
  | nlinarith; done
  | apply sq_nonneg; all_goals attack; done
  | apply div_nonneg; all_goals attack; done
  | field_simp; ring; done
  | linarith; done
)

macro "obvious": tactic =>`(tactic|
  first
  | attack; done
  | congr; all_goals attack; done
  | gcongr; all_goals attack; done
  | fail "Could not prove this goal automatically. It might not be as obvious as you think!"
)

macro "in_set" : term => `(true)



macro "term_trivial": tactic =>`(tactic|
  first
  | simp; done
  | ring; done
  | ring_nf; done
  | linarith; done
  | fail "Could not prove this goal automatically. Afterall, this is an ad hoc implementation."
)

macro "old_main_hypothesis": tactic =>`(tactic|
  first
  | assumption; done
  | fail "Could not prove this goal automatically. Afterall, this is an ad hoc implementation."
)

macro "let_assigned": tactic =>`(tactic|
  first
  | dsimp; done
  | fail "Could not prove this goal automatically. Afterall, this is an ad hoc implementation."
)

macro "term_equivalence": tactic =>`(tactic|
  first
  | simp; done
  | ring; done
  | ring_nf; done
  | linarith; done
  | fail "Could not prove this goal automatically. Afterall, this is an ad hoc implementation."
)

macro "comm_ring": tactic =>`(tactic|
  first
  | ring; done
  | ring_nf; done
  | fail "Could not prove this goal automatically. Afterall, this is an ad hoc implementation."
)

macro "litnum_reduce": tactic =>`(tactic|
  first
  | simp; done
  | simp [*]; done
  | fail "Could not prove this goal automatically. Afterall, this is an ad hoc implementation."
)

macro "litnum_bound": tactic =>`(tactic|
  first
  | linarith; done
  | fail "Could not prove this goal automatically. Afterall, this is an ad hoc implementation."
)

macro "term_derivation_variable": tactic =>`(tactic|
  first
  | rfl; done
  | fail "Could not prove this goal automatically. Afterall, this is an ad hoc implementation."
)

macro "term_derivation_literal": tactic =>`(tactic|
  first
  | rfl; done
  | fail "Could not prove this goal automatically. Afterall, this is an ad hoc implementation."
)

macro "term_derivation_item_path": tactic =>`(tactic|
  first
  | rfl; done
  | fail "Could not prove this goal automatically. Afterall, this is an ad hoc implementation."
)

macro "term_derivation_sum": tactic =>`(tactic|
  first
  | rfl; done
  | ring; done
  | fail "Could not prove this goal automatically. Afterall, this is an ad hoc implementation."
)

macro "term_derivation_sub": tactic =>`(tactic|
  first
  | rfl; done
  | ring; done
  | fail "Could not prove this goal automatically. Afterall, this is an ad hoc implementation."
)

macro "term_derivation_product": tactic =>`(tactic|
  first
  | rfl; done
  | ring; done
  | fail "Could not prove this goal automatically. Afterall, this is an ad hoc implementation."
)

macro "term_derivation_div": tactic =>`(tactic|
  first
  | rfl; done
  | ring; done
  | fail "Could not prove this goal automatically. Afterall, this is an ad hoc implementation."
)

macro "term_derivation_finalize" d1:term:1024 d2:term:1024 : tactic =>`(tactic|
  (apply (Iff.mpr $d2); apply (Iff.mp $d1); assumption)
)

macro "term_derivation_chaining_separated_list": tactic =>`(tactic|
  first
  | ring; done
  | (constructor; repeat (intro h; linarith))
  | fail "Could not prove this goal automatically. Afterall, this is an ad hoc implementation."
)

macro "term_derivation_square": tactic =>`(tactic|
  first
  | ring; done
  | fail "Could not prove this goal automatically. Afterall, this is an ad hoc implementation."
)

macro "term_derivation_power": tactic =>`(tactic|
  first
  | rfl; done
  | ring; done
  | fail "Could not prove this goal automatically. Afterall, this is an ad hoc implementation."
)

def h : 1 + 1 = 2 := by
  have h1 : 1 + 1 = 2 := by term_trivial
  obvious

\end{lstlisting}
\end{tcolorbox}


\begin{example}
Problem:
\begin{tcolorbox}[colback=yellow!10, width=\linewidth]
Let $x\in\mathbb{R}$. Prove: $x^2\ge 0$.
\end{tcolorbox}

Simplified proof:
\begin{tcolorbox}[colback=blue!10, width=\linewidth]
The square of a real number is non-negative.
\end{tcolorbox}
\end{example}

Elaborated proof:
\begin{tcolorbox}[colback=green!10, width=\linewidth]
The square of a real number is non-negative.
\end{tcolorbox}

Regularized proof:
\begin{tcolorbox}[colback=red!10, width=\linewidth]
Let $x\in\mathbb{R}$.
The goal is to prove ${{x}}^2 \ge 0$.
We have ${{x}}^2 \ge 0$.
\end{tcolorbox}

Lean 4 code:
\begin{tcolorbox}[colback=white!10, width=\linewidth]
\begin{lstlisting}[language=Lean4]
import Mathlib
syntax "attack" : tactic

macro_rules
| `(tactic| attack) => `(tactic|
  first
  | simp; done
  | ring; done
  | ring_nf; rw [Real.sq_sqrt]; ring; all_goals attack; done
  | nlinarith; done
  | apply sq_nonneg; all_goals attack; done
  | apply div_nonneg; all_goals attack; done
  | field_simp; ring; done
  | linarith; done
)

macro "obvious": tactic =>`(tactic|
  first
  | attack; done
  | congr; all_goals attack; done
  | gcongr; all_goals attack; done
  | fail "Could not prove this goal automatically. It might not be as obvious as you think!"
)

macro "in_set" : term => `(true)



macro "term_trivial": tactic =>`(tactic|
  first
  | simp; done
  | ring; done
  | ring_nf; done
  | linarith; done
  | fail "Could not prove this goal automatically. Afterall, this is an ad hoc implementation."
)

macro "old_main_hypothesis": tactic =>`(tactic|
  first
  | assumption; done
  | fail "Could not prove this goal automatically. Afterall, this is an ad hoc implementation."
)

macro "let_assigned": tactic =>`(tactic|
  first
  | dsimp; done
  | fail "Could not prove this goal automatically. Afterall, this is an ad hoc implementation."
)

macro "term_equivalence": tactic =>`(tactic|
  first
  | simp; done
  | ring; done
  | ring_nf; done
  | linarith; done
  | fail "Could not prove this goal automatically. Afterall, this is an ad hoc implementation."
)

macro "comm_ring": tactic =>`(tactic|
  first
  | ring; done
  | ring_nf; done
  | fail "Could not prove this goal automatically. Afterall, this is an ad hoc implementation."
)

macro "litnum_reduce": tactic =>`(tactic|
  first
  | simp; done
  | simp [*]; done
  | fail "Could not prove this goal automatically. Afterall, this is an ad hoc implementation."
)

macro "litnum_bound": tactic =>`(tactic|
  first
  | linarith; done
  | fail "Could not prove this goal automatically. Afterall, this is an ad hoc implementation."
)

macro "term_derivation_variable": tactic =>`(tactic|
  first
  | rfl; done
  | fail "Could not prove this goal automatically. Afterall, this is an ad hoc implementation."
)

macro "term_derivation_literal": tactic =>`(tactic|
  first
  | rfl; done
  | fail "Could not prove this goal automatically. Afterall, this is an ad hoc implementation."
)

macro "term_derivation_item_path": tactic =>`(tactic|
  first
  | rfl; done
  | fail "Could not prove this goal automatically. Afterall, this is an ad hoc implementation."
)

macro "term_derivation_sum": tactic =>`(tactic|
  first
  | rfl; done
  | ring; done
  | fail "Could not prove this goal automatically. Afterall, this is an ad hoc implementation."
)

macro "term_derivation_sub": tactic =>`(tactic|
  first
  | rfl; done
  | ring; done
  | fail "Could not prove this goal automatically. Afterall, this is an ad hoc implementation."
)

macro "term_derivation_product": tactic =>`(tactic|
  first
  | rfl; done
  | ring; done
  | fail "Could not prove this goal automatically. Afterall, this is an ad hoc implementation."
)

macro "term_derivation_div": tactic =>`(tactic|
  first
  | rfl; done
  | ring; done
  | fail "Could not prove this goal automatically. Afterall, this is an ad hoc implementation."
)

macro "term_derivation_finalize" d1:term:1024 d2:term:1024 : tactic =>`(tactic|
  (apply (Iff.mpr $d2); apply (Iff.mp $d1); assumption)
)

macro "term_derivation_chaining_separated_list": tactic =>`(tactic|
  first
  | ring; done
  | (constructor; repeat (intro h; linarith))
  | fail "Could not prove this goal automatically. Afterall, this is an ad hoc implementation."
)

macro "term_derivation_square": tactic =>`(tactic|
  first
  | ring; done
  | fail "Could not prove this goal automatically. Afterall, this is an ad hoc implementation."
)

macro "term_derivation_power": tactic =>`(tactic|
  first
  | rfl; done
  | ring; done
  | fail "Could not prove this goal automatically. Afterall, this is an ad hoc implementation."
)

def h (x : ℝ) : x ^ 2 ≥ (0 : ℝ) := by
  have h1 : x ^ 2 ≥ (0 : ℝ) := by apply square_nonnegative
  obvious

\end{lstlisting}
\end{tcolorbox}


\begin{example}
Problem:
\begin{tcolorbox}[colback=yellow!10, width=\linewidth]
Let $a\in\mathbb{R}$. Let $b\in\mathbb{R}$. Prove: $a+b=b+a$.
\end{tcolorbox}

Simplified proof:
\begin{tcolorbox}[colback=blue!10, width=\linewidth]
Trivial by the commutativity of addition for real numbers.
\end{tcolorbox}
\end{example}

Elaborated proof:
\begin{tcolorbox}[colback=green!10, width=\linewidth]
Trivial by the commutativity of addition for real numbers.
\end{tcolorbox}

Regularized proof:
\begin{tcolorbox}[colback=red!10, width=\linewidth]
Let $a\in\mathbb{R}$.
Let $b\in\mathbb{R}$.
The goal is to prove $a+b=b+a$.
We have $a+b=b+a$ by the commutativity of addition for real numbers.
\end{tcolorbox}

Lean 4 code:
\begin{tcolorbox}[colback=white!10, width=\linewidth]
\begin{lstlisting}[language=Lean4]
import Mathlib
syntax "attack" : tactic

macro_rules
| `(tactic| attack) => `(tactic|
  first
  | simp; done
  | ring; done
  | ring_nf; rw [Real.sq_sqrt]; ring; all_goals attack; done
  | nlinarith; done
  | apply sq_nonneg; all_goals attack; done
  | apply div_nonneg; all_goals attack; done
  | field_simp; ring; done
  | linarith; done
)

macro "obvious": tactic =>`(tactic|
  first
  | attack; done
  | congr; all_goals attack; done
  | gcongr; all_goals attack; done
  | fail "Could not prove this goal automatically. It might not be as obvious as you think!"
)

macro "in_set" : term => `(true)



macro "term_trivial": tactic =>`(tactic|
  first
  | simp; done
  | ring; done
  | ring_nf; done
  | linarith; done
  | fail "Could not prove this goal automatically. Afterall, this is an ad hoc implementation."
)

macro "old_main_hypothesis": tactic =>`(tactic|
  first
  | assumption; done
  | fail "Could not prove this goal automatically. Afterall, this is an ad hoc implementation."
)

macro "let_assigned": tactic =>`(tactic|
  first
  | dsimp; done
  | fail "Could not prove this goal automatically. Afterall, this is an ad hoc implementation."
)

macro "term_equivalence": tactic =>`(tactic|
  first
  | simp; done
  | ring; done
  | ring_nf; done
  | linarith; done
  | fail "Could not prove this goal automatically. Afterall, this is an ad hoc implementation."
)

macro "comm_ring": tactic =>`(tactic|
  first
  | ring; done
  | ring_nf; done
  | fail "Could not prove this goal automatically. Afterall, this is an ad hoc implementation."
)

macro "litnum_reduce": tactic =>`(tactic|
  first
  | simp; done
  | simp [*]; done
  | fail "Could not prove this goal automatically. Afterall, this is an ad hoc implementation."
)

macro "litnum_bound": tactic =>`(tactic|
  first
  | linarith; done
  | fail "Could not prove this goal automatically. Afterall, this is an ad hoc implementation."
)

macro "term_derivation_variable": tactic =>`(tactic|
  first
  | rfl; done
  | fail "Could not prove this goal automatically. Afterall, this is an ad hoc implementation."
)

macro "term_derivation_literal": tactic =>`(tactic|
  first
  | rfl; done
  | fail "Could not prove this goal automatically. Afterall, this is an ad hoc implementation."
)

macro "term_derivation_item_path": tactic =>`(tactic|
  first
  | rfl; done
  | fail "Could not prove this goal automatically. Afterall, this is an ad hoc implementation."
)

macro "term_derivation_sum": tactic =>`(tactic|
  first
  | rfl; done
  | ring; done
  | fail "Could not prove this goal automatically. Afterall, this is an ad hoc implementation."
)

macro "term_derivation_sub": tactic =>`(tactic|
  first
  | rfl; done
  | ring; done
  | fail "Could not prove this goal automatically. Afterall, this is an ad hoc implementation."
)

macro "term_derivation_product": tactic =>`(tactic|
  first
  | rfl; done
  | ring; done
  | fail "Could not prove this goal automatically. Afterall, this is an ad hoc implementation."
)

macro "term_derivation_div": tactic =>`(tactic|
  first
  | rfl; done
  | ring; done
  | fail "Could not prove this goal automatically. Afterall, this is an ad hoc implementation."
)

macro "term_derivation_finalize" d1:term:1024 d2:term:1024 : tactic =>`(tactic|
  (apply (Iff.mpr $d2); apply (Iff.mp $d1); assumption)
)

macro "term_derivation_chaining_separated_list": tactic =>`(tactic|
  first
  | ring; done
  | (constructor; repeat (intro h; linarith))
  | fail "Could not prove this goal automatically. Afterall, this is an ad hoc implementation."
)

macro "term_derivation_square": tactic =>`(tactic|
  first
  | ring; done
  | fail "Could not prove this goal automatically. Afterall, this is an ad hoc implementation."
)

macro "term_derivation_power": tactic =>`(tactic|
  first
  | rfl; done
  | ring; done
  | fail "Could not prove this goal automatically. Afterall, this is an ad hoc implementation."
)

def h (a b : ℝ) : a + b = b + a := by
  have h1 : a + b = b + a := by term_trivial
  obvious

\end{lstlisting}
\end{tcolorbox}


\begin{example}
Problem:
\begin{tcolorbox}[colback=yellow!10, width=\linewidth]
Let $x\in\mathbb{R}$. Assume $x> 0$. Prove: $x + \frac{1}{x} \ge 2$.
\end{tcolorbox}

Simplified proof:
\begin{tcolorbox}[colback=blue!10, width=\linewidth]
Since $x>0$, we have $(x-1)^2 \ge 0$. Thus, $x - 2 + \frac{1}{x} \ge 0$, so $x + \frac{1}{x} \ge 2$.
\end{tcolorbox}
\end{example}

Elaborated proof:
\begin{tcolorbox}[colback=green!10, width=\linewidth]
Since $x>0$, we have $(x-1)^2 = x^2 - 2x + 1 \ge 0$. Thus, $x^2 - 2x + 1 \ge 0$, so $x^2 + 1 \ge 2x$. Since $x > 0$, dividing both sides by $x$ gives $x + \frac{1}{x} \ge 2$.
\end{tcolorbox}

Regularized proof:
\begin{tcolorbox}[colback=red!10, width=\linewidth]
Let $x\in\mathbb{R}$.
Assume $x>0$.
The goal is to prove $x + \frac{1}{x} \ge 2$.
We have $x>0$.
We have ${{(x-1)}}^2 = x^2 - 2x + 1 \ge 0$.
We have $x^2 - 2x + 1 \ge 0$.
We have $x^2 + 1 \ge 2x$.
We have $x > 0$.
We have $x + \frac{1}{x} \ge 2$.
\end{tcolorbox}

Lean 4 code:
\begin{tcolorbox}[colback=white!10, width=\linewidth]
\begin{lstlisting}[language=Lean4]
import Mathlib
syntax "attack" : tactic

macro_rules
| `(tactic| attack) => `(tactic|
  first
  | simp; done
  | ring; done
  | ring_nf; rw [Real.sq_sqrt]; ring; all_goals attack; done
  | nlinarith; done
  | apply sq_nonneg; all_goals attack; done
  | apply div_nonneg; all_goals attack; done
  | field_simp; ring; done
  | linarith; done
)

macro "obvious": tactic =>`(tactic|
  first
  | attack; done
  | congr; all_goals attack; done
  | gcongr; all_goals attack; done
  | fail "Could not prove this goal automatically. It might not be as obvious as you think!"
)

macro "in_set" : term => `(true)



macro "term_trivial": tactic =>`(tactic|
  first
  | simp; done
  | ring; done
  | ring_nf; done
  | linarith; done
  | fail "Could not prove this goal automatically. Afterall, this is an ad hoc implementation."
)

macro "old_main_hypothesis": tactic =>`(tactic|
  first
  | assumption; done
  | fail "Could not prove this goal automatically. Afterall, this is an ad hoc implementation."
)

macro "let_assigned": tactic =>`(tactic|
  first
  | dsimp; done
  | fail "Could not prove this goal automatically. Afterall, this is an ad hoc implementation."
)

macro "term_equivalence": tactic =>`(tactic|
  first
  | simp; done
  | ring; done
  | ring_nf; done
  | linarith; done
  | fail "Could not prove this goal automatically. Afterall, this is an ad hoc implementation."
)

macro "comm_ring": tactic =>`(tactic|
  first
  | ring; done
  | ring_nf; done
  | fail "Could not prove this goal automatically. Afterall, this is an ad hoc implementation."
)

macro "litnum_reduce": tactic =>`(tactic|
  first
  | simp; done
  | simp [*]; done
  | fail "Could not prove this goal automatically. Afterall, this is an ad hoc implementation."
)

macro "litnum_bound": tactic =>`(tactic|
  first
  | linarith; done
  | fail "Could not prove this goal automatically. Afterall, this is an ad hoc implementation."
)

macro "term_derivation_variable": tactic =>`(tactic|
  first
  | rfl; done
  | fail "Could not prove this goal automatically. Afterall, this is an ad hoc implementation."
)

macro "term_derivation_literal": tactic =>`(tactic|
  first
  | rfl; done
  | fail "Could not prove this goal automatically. Afterall, this is an ad hoc implementation."
)

macro "term_derivation_item_path": tactic =>`(tactic|
  first
  | rfl; done
  | fail "Could not prove this goal automatically. Afterall, this is an ad hoc implementation."
)

macro "term_derivation_sum": tactic =>`(tactic|
  first
  | rfl; done
  | ring; done
  | fail "Could not prove this goal automatically. Afterall, this is an ad hoc implementation."
)

macro "term_derivation_sub": tactic =>`(tactic|
  first
  | rfl; done
  | ring; done
  | fail "Could not prove this goal automatically. Afterall, this is an ad hoc implementation."
)

macro "term_derivation_product": tactic =>`(tactic|
  first
  | rfl; done
  | ring; done
  | fail "Could not prove this goal automatically. Afterall, this is an ad hoc implementation."
)

macro "term_derivation_div": tactic =>`(tactic|
  first
  | rfl; done
  | ring; done
  | fail "Could not prove this goal automatically. Afterall, this is an ad hoc implementation."
)

macro "term_derivation_finalize" d1:term:1024 d2:term:1024 : tactic =>`(tactic|
  (apply (Iff.mpr $d2); apply (Iff.mp $d1); assumption)
)

macro "term_derivation_chaining_separated_list": tactic =>`(tactic|
  first
  | ring; done
  | (constructor; repeat (intro h; linarith))
  | fail "Could not prove this goal automatically. Afterall, this is an ad hoc implementation."
)

macro "term_derivation_square": tactic =>`(tactic|
  first
  | ring; done
  | fail "Could not prove this goal automatically. Afterall, this is an ad hoc implementation."
)

macro "term_derivation_power": tactic =>`(tactic|
  first
  | rfl; done
  | ring; done
  | fail "Could not prove this goal automatically. Afterall, this is an ad hoc implementation."
)

def h (x : ℝ) (h1 : x > (0 : ℝ)) : x + (1 : ℝ) / x ≥ (2 : ℝ) := by
  have h2 : x > (0 : ℝ) := by old_main_hypothesis
  first
  | have h3 : (x - (1 : ℝ)) ^ 2 ≥ (0 : ℝ) := by calc
    (x - (1 : ℝ)) ^ 2 = x ^ 2 - (2 : ℝ) * x + (1 : ℝ) := by obvious
    _ ≥ (0 : ℝ) := by obvious
  | have h4 : x ^ 2 - (2 : ℝ) * x + (1 : ℝ) ≥ (0 : ℝ) := by calc
    x ^ 2 - (2 : ℝ) * x + (1 : ℝ) = (x - (1 : ℝ)) ^ 2 := by obvious
    _ ≥ (0 : ℝ) := by obvious
  have h5 : x ^ 2 - (2 : ℝ) * x + (1 : ℝ) ≥ (0 : ℝ) := by obvious
  have h6 : x ^ 2 + (1 : ℝ) ≥ (2 : ℝ) * x := by
    have d : x = x := by term_derivation_variable
    have d1 : 2 = 2 := by term_derivation_literal
    have d2 : x ^ 2 = x ^ 2 := by term_derivation_power
    have d3 : (2 : ℝ) = 2 := by term_derivation_literal
    have d4 : x = x := by term_derivation_variable
    have d5 : (2 : ℝ) * x = (2 : ℝ) * x := by term_derivation_product
    have d6 : x ^ 2 - (2 : ℝ) * x = (-2 : ℝ) * x + x ^ 2 := by term_derivation_sub
    have d7 : (1 : ℝ) = 1 := by term_derivation_literal
    have d8 : x ^ 2 - (2 : ℝ) * x + (1 : ℝ) = (1 : ℝ) + (-2 : ℝ) * x + x ^ 2 := by term_derivation_sum
    have d9 : (0 : ℝ) = 0 := by term_derivation_literal
    have d10 : x ^ 2 - (2 : ℝ) * x + (1 : ℝ) ≥ (0 : ℝ) ↔ (1 : ℝ) + (-2 : ℝ) * x + x ^ 2 ≥ (0 : ℝ) := by term_derivation_chaining_separated_list
    have d11 : x = x := by term_derivation_variable
    have d12 : 2 = 2 := by term_derivation_literal
    have d13 : x ^ 2 = x ^ 2 := by term_derivation_power
    have d14 : (1 : ℝ) = 1 := by term_derivation_literal
    have d15 : x ^ 2 + (1 : ℝ) = (1 : ℝ) + x ^ 2 := by term_derivation_sum
    have d16 : (2 : ℝ) = 2 := by term_derivation_literal
    have d17 : x = x := by term_derivation_variable
    have d18 : (2 : ℝ) * x = (2 : ℝ) * x := by term_derivation_product
    have d19 : x ^ 2 + (1 : ℝ) ≥ (2 : ℝ) * x ↔ (1 : ℝ) + (-2 : ℝ) * x + x ^ 2 ≥ (0 : ℝ) := by term_derivation_chaining_separated_list
    have d20 : x ^ 2 + (1 : ℝ) ≥ (2 : ℝ) * x := by term_derivation_finalize
    term_derivation_finalize
  have h7 : x > (0 : ℝ) := by old_main_hypothesis
  have h8 : x + (1 : ℝ) / x ≥ (2 : ℝ) := by obvious
  obvious

\end{lstlisting}
\end{tcolorbox}


\begin{example}
Problem:
\begin{tcolorbox}[colback=yellow!10, width=\linewidth]
Let $x\in\mathbb{R}$. Prove: $x^2 + 1\ge 2x$.
\end{tcolorbox}

Simplified proof:
\begin{tcolorbox}[colback=blue!10, width=\linewidth]
$(x-1)^2 \ge 0$, so $x^2 + 1 \ge 2x$.
\end{tcolorbox}
\end{example}

Elaborated proof:
\begin{tcolorbox}[colback=green!10, width=\linewidth]
$(x-1)^2 = x^2 - 2x + 1 \ge 0$, so $x^2 + 1 \ge 2x$.
\end{tcolorbox}

Regularized proof:
\begin{tcolorbox}[colback=red!10, width=\linewidth]
Let $x\in\mathbb{R}$.
The goal is to prove $x^2 + 1 \ge 2x$.
We have ${{x-1}}^2 = x^2 - 2x + 1 \ge 0$.
We have $x^2 + 1 \ge 2x$.
\end{tcolorbox}

Lean 4 code:
\begin{tcolorbox}[colback=white!10, width=\linewidth]
\begin{lstlisting}[language=Lean4]
import Mathlib
syntax "attack" : tactic

macro_rules
| `(tactic| attack) => `(tactic|
  first
  | simp; done
  | ring; done
  | ring_nf; rw [Real.sq_sqrt]; ring; all_goals attack; done
  | nlinarith; done
  | apply sq_nonneg; all_goals attack; done
  | apply div_nonneg; all_goals attack; done
  | field_simp; ring; done
  | linarith; done
)

macro "obvious": tactic =>`(tactic|
  first
  | attack; done
  | congr; all_goals attack; done
  | gcongr; all_goals attack; done
  | fail "Could not prove this goal automatically. It might not be as obvious as you think!"
)

macro "in_set" : term => `(true)



macro "term_trivial": tactic =>`(tactic|
  first
  | simp; done
  | ring; done
  | ring_nf; done
  | linarith; done
  | fail "Could not prove this goal automatically. Afterall, this is an ad hoc implementation."
)

macro "old_main_hypothesis": tactic =>`(tactic|
  first
  | assumption; done
  | fail "Could not prove this goal automatically. Afterall, this is an ad hoc implementation."
)

macro "let_assigned": tactic =>`(tactic|
  first
  | dsimp; done
  | fail "Could not prove this goal automatically. Afterall, this is an ad hoc implementation."
)

macro "term_equivalence": tactic =>`(tactic|
  first
  | simp; done
  | ring; done
  | ring_nf; done
  | linarith; done
  | fail "Could not prove this goal automatically. Afterall, this is an ad hoc implementation."
)

macro "comm_ring": tactic =>`(tactic|
  first
  | ring; done
  | ring_nf; done
  | fail "Could not prove this goal automatically. Afterall, this is an ad hoc implementation."
)

macro "litnum_reduce": tactic =>`(tactic|
  first
  | simp; done
  | simp [*]; done
  | fail "Could not prove this goal automatically. Afterall, this is an ad hoc implementation."
)

macro "litnum_bound": tactic =>`(tactic|
  first
  | linarith; done
  | fail "Could not prove this goal automatically. Afterall, this is an ad hoc implementation."
)

macro "term_derivation_variable": tactic =>`(tactic|
  first
  | rfl; done
  | fail "Could not prove this goal automatically. Afterall, this is an ad hoc implementation."
)

macro "term_derivation_literal": tactic =>`(tactic|
  first
  | rfl; done
  | fail "Could not prove this goal automatically. Afterall, this is an ad hoc implementation."
)

macro "term_derivation_item_path": tactic =>`(tactic|
  first
  | rfl; done
  | fail "Could not prove this goal automatically. Afterall, this is an ad hoc implementation."
)

macro "term_derivation_sum": tactic =>`(tactic|
  first
  | rfl; done
  | ring; done
  | fail "Could not prove this goal automatically. Afterall, this is an ad hoc implementation."
)

macro "term_derivation_sub": tactic =>`(tactic|
  first
  | rfl; done
  | ring; done
  | fail "Could not prove this goal automatically. Afterall, this is an ad hoc implementation."
)

macro "term_derivation_product": tactic =>`(tactic|
  first
  | rfl; done
  | ring; done
  | fail "Could not prove this goal automatically. Afterall, this is an ad hoc implementation."
)

macro "term_derivation_div": tactic =>`(tactic|
  first
  | rfl; done
  | ring; done
  | fail "Could not prove this goal automatically. Afterall, this is an ad hoc implementation."
)

macro "term_derivation_finalize" d1:term:1024 d2:term:1024 : tactic =>`(tactic|
  (apply (Iff.mpr $d2); apply (Iff.mp $d1); assumption)
)

macro "term_derivation_chaining_separated_list": tactic =>`(tactic|
  first
  | ring; done
  | (constructor; repeat (intro h; linarith))
  | fail "Could not prove this goal automatically. Afterall, this is an ad hoc implementation."
)

macro "term_derivation_square": tactic =>`(tactic|
  first
  | ring; done
  | fail "Could not prove this goal automatically. Afterall, this is an ad hoc implementation."
)

macro "term_derivation_power": tactic =>`(tactic|
  first
  | rfl; done
  | ring; done
  | fail "Could not prove this goal automatically. Afterall, this is an ad hoc implementation."
)

def h (x : ℝ) : x ^ 2 + (1 : ℝ) ≥ (2 : ℝ) * x := by
  first
  | have h1 : (x - (1 : ℝ)) ^ 2 ≥ (0 : ℝ) := by calc
    (x - (1 : ℝ)) ^ 2 = x ^ 2 - (2 : ℝ) * x + (1 : ℝ) := by obvious
    _ ≥ (0 : ℝ) := by obvious
  | have h2 : x ^ 2 - (2 : ℝ) * x + (1 : ℝ) ≥ (0 : ℝ) := by calc
    x ^ 2 - (2 : ℝ) * x + (1 : ℝ) = (x - (1 : ℝ)) ^ 2 := by obvious
    _ ≥ (0 : ℝ) := by obvious
  have h3 : x ^ 2 + (1 : ℝ) ≥ (2 : ℝ) * x := by obvious
  obvious

\end{lstlisting}
\end{tcolorbox}


\begin{example}
Problem:
\begin{tcolorbox}[colback=yellow!10, width=\linewidth]
Let $x\in\mathbb{R}$. Prove: $x + 1 \ge 2\sqrt{x}$.
\end{tcolorbox}

Simplified proof:
\begin{tcolorbox}[colback=blue!10, width=\linewidth]
We have $(\sqrt{x}-1)^2 \ge 0$, so $x+1 \ge 2\sqrt{x}$.
\end{tcolorbox}
\end{example}

Elaborated proof:
\begin{tcolorbox}[colback=green!10, width=\linewidth]
We have $(\sqrt{x}-1)^2 = x - 2\sqrt{x} + 1 \ge 0$, so $x+1 \ge 2\sqrt{x}$.
\end{tcolorbox}

Regularized proof:
\begin{tcolorbox}[colback=red!10, width=\linewidth]
Let $x\in\mathbb{R}$.
Assume $x\ge 0$.
The goal is to prove $x+1 \ge 2\sqrt{x}$.
We have ${(\sqrt{x}-1)}^2 = x - 2\sqrt{x} + 1 \ge 0$.
We have $x+1 \ge 2\sqrt{x}$.
\end{tcolorbox}

Lean 4 code:
\begin{tcolorbox}[colback=white!10, width=\linewidth]
\begin{lstlisting}[language=Lean4]
import Mathlib
syntax "attack" : tactic

macro_rules
| `(tactic| attack) => `(tactic|
  first
  | simp; done
  | ring; done
  | ring_nf; rw [Real.sq_sqrt]; ring; all_goals attack; done
  | nlinarith; done
  | apply sq_nonneg; all_goals attack; done
  | apply div_nonneg; all_goals attack; done
  | field_simp; ring; done
  | linarith; done
)

macro "obvious": tactic =>`(tactic|
  first
  | attack; done
  | congr; all_goals attack; done
  | gcongr; all_goals attack; done
  | fail "Could not prove this goal automatically. It might not be as obvious as you think!"
)

macro "in_set" : term => `(true)



macro "term_trivial": tactic =>`(tactic|
  first
  | simp; done
  | ring; done
  | ring_nf; done
  | linarith; done
  | fail "Could not prove this goal automatically. Afterall, this is an ad hoc implementation."
)

macro "old_main_hypothesis": tactic =>`(tactic|
  first
  | assumption; done
  | fail "Could not prove this goal automatically. Afterall, this is an ad hoc implementation."
)

macro "let_assigned": tactic =>`(tactic|
  first
  | dsimp; done
  | fail "Could not prove this goal automatically. Afterall, this is an ad hoc implementation."
)

macro "term_equivalence": tactic =>`(tactic|
  first
  | simp; done
  | ring; done
  | ring_nf; done
  | linarith; done
  | fail "Could not prove this goal automatically. Afterall, this is an ad hoc implementation."
)

macro "comm_ring": tactic =>`(tactic|
  first
  | ring; done
  | ring_nf; done
  | fail "Could not prove this goal automatically. Afterall, this is an ad hoc implementation."
)

macro "litnum_reduce": tactic =>`(tactic|
  first
  | simp; done
  | simp [*]; done
  | fail "Could not prove this goal automatically. Afterall, this is an ad hoc implementation."
)

macro "litnum_bound": tactic =>`(tactic|
  first
  | linarith; done
  | fail "Could not prove this goal automatically. Afterall, this is an ad hoc implementation."
)

macro "term_derivation_variable": tactic =>`(tactic|
  first
  | rfl; done
  | fail "Could not prove this goal automatically. Afterall, this is an ad hoc implementation."
)

macro "term_derivation_literal": tactic =>`(tactic|
  first
  | rfl; done
  | fail "Could not prove this goal automatically. Afterall, this is an ad hoc implementation."
)

macro "term_derivation_item_path": tactic =>`(tactic|
  first
  | rfl; done
  | fail "Could not prove this goal automatically. Afterall, this is an ad hoc implementation."
)

macro "term_derivation_sum": tactic =>`(tactic|
  first
  | rfl; done
  | ring; done
  | fail "Could not prove this goal automatically. Afterall, this is an ad hoc implementation."
)

macro "term_derivation_sub": tactic =>`(tactic|
  first
  | rfl; done
  | ring; done
  | fail "Could not prove this goal automatically. Afterall, this is an ad hoc implementation."
)

macro "term_derivation_product": tactic =>`(tactic|
  first
  | rfl; done
  | ring; done
  | fail "Could not prove this goal automatically. Afterall, this is an ad hoc implementation."
)

macro "term_derivation_div": tactic =>`(tactic|
  first
  | rfl; done
  | ring; done
  | fail "Could not prove this goal automatically. Afterall, this is an ad hoc implementation."
)

macro "term_derivation_finalize" d1:term:1024 d2:term:1024 : tactic =>`(tactic|
  (apply (Iff.mpr $d2); apply (Iff.mp $d1); assumption)
)

macro "term_derivation_chaining_separated_list": tactic =>`(tactic|
  first
  | ring; done
  | (constructor; repeat (intro h; linarith))
  | fail "Could not prove this goal automatically. Afterall, this is an ad hoc implementation."
)

macro "term_derivation_square": tactic =>`(tactic|
  first
  | ring; done
  | fail "Could not prove this goal automatically. Afterall, this is an ad hoc implementation."
)

macro "term_derivation_power": tactic =>`(tactic|
  first
  | rfl; done
  | ring; done
  | fail "Could not prove this goal automatically. Afterall, this is an ad hoc implementation."
)

def h (x : ℝ) (h1 : x ≥ (0 : ℝ)) : x + (1 : ℝ) ≥ (2 : ℝ) * √ x := by
  first
  | have h2 : (√ x - (1 : ℝ)) ^ 2 ≥ (0 : ℝ) := by calc
    (√ x - (1 : ℝ)) ^ 2 = x - (2 : ℝ) * √ x + (1 : ℝ) := by obvious
    _ ≥ (0 : ℝ) := by obvious
  | have h3 : x - (2 : ℝ) * √ x + (1 : ℝ) ≥ (0 : ℝ) := by calc
    x - (2 : ℝ) * √ x + (1 : ℝ) = (√ x - (1 : ℝ)) ^ 2 := by obvious
    _ ≥ (0 : ℝ) := by obvious
  have h4 : x + (1 : ℝ) ≥ (2 : ℝ) * √ x := by obvious
  obvious

\end{lstlisting}
\end{tcolorbox}


\begin{example}
Problem:
\begin{tcolorbox}[colback=yellow!10, width=\linewidth]
Let $x\in\mathbb{R}$. Assume $x>0$.
    Let $y\in\mathbb{R}$. Assume $y>0$.
    Prove: $\frac{1}{x} + \frac{1}{y} \ge \frac{4}{x+y}$.
\end{tcolorbox}

Simplified proof:
\begin{tcolorbox}[colback=blue!10, width=\linewidth]
Since $x>0$ and $y>0$, we have $(x-y)^2 \ge 0$. Thus $(x+y)^2 \ge 4xy$. Dividing by $xy(x+y)$ gives $\frac{1}{x}+\frac{1}{y} \ge \frac{4}{x+y}$.
\end{tcolorbox}
\end{example}

Elaborated proof:
\begin{tcolorbox}[colback=green!10, width=\linewidth]
Since $x>0$ and $y>0$, we have $(x-y)^2 \ge 0$. Thus $(x-y)^2 = x^2 -2xy + y^2 \ge 0$, so $x^2 + y^2 \ge 2xy$, and hence $x^2 + 2xy + y^2 = (x+y)^2 \ge 4xy$. Dividing by $xy(x+y)$ gives $\frac{(x+y)^2}{xy(x+y)} = \frac{x+y}{xy} = \frac{x}{xy} + \frac{y}{xy} = \frac{1}{y}+\frac{1}{x} \ge \frac{4xy}{xy(x+y)} = \frac{4}{x+y}$.
\end{tcolorbox}

Regularized proof:
\begin{tcolorbox}[colback=red!10, width=\linewidth]
Let $x\in\mathbb{R}$.
Assume $x>0$.
Let $y\in\mathbb{R}$.
Assume $y>0$.
The goal is to prove $\frac{1}{x} + \frac{1}{y} \ge \frac{4}{x+y}$.
We have $x \in \mathbb{R}$.
We have $x>0$.
We have $y \in \mathbb{R}$.
We have $y>0$.
It's enough to show that $\frac{1}{x} + \frac{1}{y} \ge \frac{4}{x+y}$.
We have ${(x-y)}^2 \ge 0$.
We have ${(x-y)}^2 = x^2 -2xy + y^2 \ge 0$.
We have $x^2 + y^2 \ge 2xy$.
We have $x^2 + 2xy + y^2 = {(x+y)}^2 \ge 4xy$.
We have $\frac{{(x+y)}^2}{xy(x+y)} = \frac{x+y}{xy} = \frac{x}{xy} + \frac{y}{xy} = \frac{1}{y}+\frac{1}{x} \ge \frac{4xy}{xy(x+y)} = \frac{4}{x+y}$.
\end{tcolorbox}

Lean 4 code:
\begin{tcolorbox}[colback=white!10, width=\linewidth]
\begin{lstlisting}[language=Lean4]
import Mathlib
syntax "attack" : tactic

macro_rules
| `(tactic| attack) => `(tactic|
  first
  | simp; done
  | ring; done
  | ring_nf; rw [Real.sq_sqrt]; ring; all_goals attack; done
  | nlinarith; done
  | apply sq_nonneg; all_goals attack; done
  | apply div_nonneg; all_goals attack; done
  | field_simp; ring; done
  | linarith; done
)

macro "obvious": tactic =>`(tactic|
  first
  | attack; done
  | congr; all_goals attack; done
  | gcongr; all_goals attack; done
  | fail "Could not prove this goal automatically. It might not be as obvious as you think!"
)

macro "in_set" : term => `(true)



macro "term_trivial": tactic =>`(tactic|
  first
  | simp; done
  | ring; done
  | ring_nf; done
  | linarith; done
  | fail "Could not prove this goal automatically. Afterall, this is an ad hoc implementation."
)

macro "old_main_hypothesis": tactic =>`(tactic|
  first
  | assumption; done
  | fail "Could not prove this goal automatically. Afterall, this is an ad hoc implementation."
)

macro "let_assigned": tactic =>`(tactic|
  first
  | dsimp; done
  | fail "Could not prove this goal automatically. Afterall, this is an ad hoc implementation."
)

macro "term_equivalence": tactic =>`(tactic|
  first
  | simp; done
  | ring; done
  | ring_nf; done
  | linarith; done
  | fail "Could not prove this goal automatically. Afterall, this is an ad hoc implementation."
)

macro "comm_ring": tactic =>`(tactic|
  first
  | ring; done
  | ring_nf; done
  | fail "Could not prove this goal automatically. Afterall, this is an ad hoc implementation."
)

macro "litnum_reduce": tactic =>`(tactic|
  first
  | simp; done
  | simp [*]; done
  | fail "Could not prove this goal automatically. Afterall, this is an ad hoc implementation."
)

macro "litnum_bound": tactic =>`(tactic|
  first
  | linarith; done
  | fail "Could not prove this goal automatically. Afterall, this is an ad hoc implementation."
)

macro "term_derivation_variable": tactic =>`(tactic|
  first
  | rfl; done
  | fail "Could not prove this goal automatically. Afterall, this is an ad hoc implementation."
)

macro "term_derivation_literal": tactic =>`(tactic|
  first
  | rfl; done
  | fail "Could not prove this goal automatically. Afterall, this is an ad hoc implementation."
)

macro "term_derivation_item_path": tactic =>`(tactic|
  first
  | rfl; done
  | fail "Could not prove this goal automatically. Afterall, this is an ad hoc implementation."
)

macro "term_derivation_sum": tactic =>`(tactic|
  first
  | rfl; done
  | ring; done
  | fail "Could not prove this goal automatically. Afterall, this is an ad hoc implementation."
)

macro "term_derivation_sub": tactic =>`(tactic|
  first
  | rfl; done
  | ring; done
  | fail "Could not prove this goal automatically. Afterall, this is an ad hoc implementation."
)

macro "term_derivation_product": tactic =>`(tactic|
  first
  | rfl; done
  | ring; done
  | fail "Could not prove this goal automatically. Afterall, this is an ad hoc implementation."
)

macro "term_derivation_div": tactic =>`(tactic|
  first
  | rfl; done
  | ring; done
  | fail "Could not prove this goal automatically. Afterall, this is an ad hoc implementation."
)

macro "term_derivation_finalize" d1:term:1024 d2:term:1024 : tactic =>`(tactic|
  (apply (Iff.mpr $d2); apply (Iff.mp $d1); assumption)
)

macro "term_derivation_chaining_separated_list": tactic =>`(tactic|
  first
  | ring; done
  | (constructor; repeat (intro h; linarith))
  | fail "Could not prove this goal automatically. Afterall, this is an ad hoc implementation."
)

macro "term_derivation_square": tactic =>`(tactic|
  first
  | ring; done
  | fail "Could not prove this goal automatically. Afterall, this is an ad hoc implementation."
)

macro "term_derivation_power": tactic =>`(tactic|
  first
  | rfl; done
  | ring; done
  | fail "Could not prove this goal automatically. Afterall, this is an ad hoc implementation."
)

def h (x : ℝ) (h1 : x > (0 : ℝ)) (y : ℝ) (h2 : y > (0 : ℝ)) : (1 : ℝ) / x + (1 : ℝ) / y ≥ (4 : ℝ) / (x + y) := by
  have h3 : in_set := by obvious
  have h4 : x > (0 : ℝ) := by old_main_hypothesis
  have h5 : in_set := by obvious
  have h6 : y > (0 : ℝ) := by old_main_hypothesis
  have h15 : (1 : ℝ) / x + (1 : ℝ) / y ≥ (4 : ℝ) / (x + y) := by
    have h7 : (x - y) ^ 2 ≥ (0 : ℝ) := by apply square_nonnegative
    first
    | have h8 : (x - y) ^ 2 ≥ (0 : ℝ) := by calc
      (x - y) ^ 2 = x ^ 2 - (2 : ℝ) * x * y + y ^ 2 := by obvious
      _ ≥ (0 : ℝ) := by obvious
    | have h9 : x ^ 2 - (2 : ℝ) * x * y + y ^ 2 ≥ (0 : ℝ) := by calc
      x ^ 2 - (2 : ℝ) * x * y + y ^ 2 = (x - y) ^ 2 := by obvious
      _ ≥ (0 : ℝ) := by obvious
    have h10 : x ^ 2 + y ^ 2 ≥ (2 : ℝ) * x * y := by obvious
    first
    | have h11 : x ^ 2 + (2 : ℝ) * x * y + y ^ 2 ≥ (4 : ℝ) * x * y := by calc
      x ^ 2 + (2 : ℝ) * x * y + y ^ 2 = (x + y) ^ 2 := by obvious
      _ ≥ (4 : ℝ) * x * y := by obvious
    | have h12 : (x + y) ^ 2 ≥ (4 : ℝ) * x * y := by calc
      (x + y) ^ 2 = x ^ 2 + (2 : ℝ) * x * y + y ^ 2 := by obvious
      _ ≥ (4 : ℝ) * x * y := by obvious
    first
    | have h13 : (x + y) ^ 2 / (x * y * (x + y)) ≥ (4 : ℝ) / (x + y) := by calc
      (x + y) ^ 2 / (x * y * (x + y)) = (x + y) / (x * y) := by obvious
      _ = x / (x * y) + y / (x * y) := by obvious
      _ = (1 : ℝ) / y + (1 : ℝ) / x := by obvious
      _ ≥ (4 : ℝ) * x * y / (x * y * (x + y)) := by obvious
      _ = (4 : ℝ) / (x + y) := by obvious
    | have h14 : (1 : ℝ) / y + (1 : ℝ) / x ≥ (4 : ℝ) / (x + y) := by calc
      (1 : ℝ) / y + (1 : ℝ) / x = x / (x * y) + y / (x * y) := by obvious
      _ = (x + y) / (x * y) := by obvious
      _ = (x + y) ^ 2 / (x * y * (x + y)) := by obvious
      _ ≥ (4 : ℝ) * x * y / (x * y * (x + y)) := by obvious
      _ = (4 : ℝ) / (x + y) := by obvious
    obvious
  have h16 : (1 : ℝ) / x + (1 : ℝ) / y ≥ (4 : ℝ) / (x + y) := by obvious

\end{lstlisting}
\end{tcolorbox}


\begin{example}
Problem:
\begin{tcolorbox}[colback=yellow!10, width=\linewidth]
Let $a\in\mathbb{R}$. Assume $a > 0$.
    Let $b\in\mathbb{R}$. Assume $b > 0$.
    Prove: $\frac{a}{b} + \frac{b}{a} \ge 2$.
\end{tcolorbox}

Simplified proof:
\begin{tcolorbox}[colback=blue!10, width=\linewidth]
Since $a>0$ and $b>0$, we have $(\sqrt{a/b} - \sqrt{b/a})^2 \ge 0$. Thus $a/b + b/a \ge 2$.
\end{tcolorbox}
\end{example}

Elaborated proof:
\begin{tcolorbox}[colback=green!10, width=\linewidth]
Since $a>0$ and $b>0$, we have $(\sqrt{a/b} - \sqrt{b/a})^2 = (\sqrt{a/b})^2 - 2\sqrt{a/b}\sqrt{b/a} + (\sqrt{b/a})^2 = a/b - 2 + b/a \ge 0$. Thus $a/b - 2 + b/a \ge 0$, so $a/b + b/a \ge 2$.
\end{tcolorbox}

Regularized proof:
\begin{tcolorbox}[colback=red!10, width=\linewidth]
Let $a\in\mathbb{R}$.
Let $b\in\mathbb{R}$.
Assume $a>0$.
Assume $b>0$.
The goal is to prove $a/b + b/a \ge 2$.
We have $a>0$.
We have $b>0$.
We have ${{(\sqrt{a/b} - \sqrt{b/a})}}^2 = {{(\sqrt{a/b})}}^2 - 2\sqrt{a/b}\sqrt{b/a} + {{(\sqrt{b/a})}}^2 = a/b - 2 + b/a \ge 0$.
We have $a/b - 2 + b/a \ge 0$.
We have $a/b + b/a \ge 2$.
\end{tcolorbox}

Lean 4 code:
\begin{tcolorbox}[colback=white!10, width=\linewidth]
\begin{lstlisting}[language=Lean4]
import Mathlib
syntax "attack" : tactic

macro_rules
| `(tactic| attack) => `(tactic|
  first
  | simp; done
  | ring; done
  | ring_nf; rw [Real.sq_sqrt]; ring; all_goals attack; done
  | nlinarith; done
  | apply sq_nonneg; all_goals attack; done
  | apply div_nonneg; all_goals attack; done
  | field_simp; ring; done
  | linarith; done
)

macro "obvious": tactic =>`(tactic|
  first
  | attack; done
  | congr; all_goals attack; done
  | gcongr; all_goals attack; done
  | fail "Could not prove this goal automatically. It might not be as obvious as you think!"
)

macro "in_set" : term => `(true)



macro "term_trivial": tactic =>`(tactic|
  first
  | simp; done
  | ring; done
  | ring_nf; done
  | linarith; done
  | fail "Could not prove this goal automatically. Afterall, this is an ad hoc implementation."
)

macro "old_main_hypothesis": tactic =>`(tactic|
  first
  | assumption; done
  | fail "Could not prove this goal automatically. Afterall, this is an ad hoc implementation."
)

macro "let_assigned": tactic =>`(tactic|
  first
  | dsimp; done
  | fail "Could not prove this goal automatically. Afterall, this is an ad hoc implementation."
)

macro "term_equivalence": tactic =>`(tactic|
  first
  | simp; done
  | ring; done
  | ring_nf; done
  | linarith; done
  | fail "Could not prove this goal automatically. Afterall, this is an ad hoc implementation."
)

macro "comm_ring": tactic =>`(tactic|
  first
  | ring; done
  | ring_nf; done
  | fail "Could not prove this goal automatically. Afterall, this is an ad hoc implementation."
)

macro "litnum_reduce": tactic =>`(tactic|
  first
  | simp; done
  | simp [*]; done
  | fail "Could not prove this goal automatically. Afterall, this is an ad hoc implementation."
)

macro "litnum_bound": tactic =>`(tactic|
  first
  | linarith; done
  | fail "Could not prove this goal automatically. Afterall, this is an ad hoc implementation."
)

macro "term_derivation_variable": tactic =>`(tactic|
  first
  | rfl; done
  | fail "Could not prove this goal automatically. Afterall, this is an ad hoc implementation."
)

macro "term_derivation_literal": tactic =>`(tactic|
  first
  | rfl; done
  | fail "Could not prove this goal automatically. Afterall, this is an ad hoc implementation."
)

macro "term_derivation_item_path": tactic =>`(tactic|
  first
  | rfl; done
  | fail "Could not prove this goal automatically. Afterall, this is an ad hoc implementation."
)

macro "term_derivation_sum": tactic =>`(tactic|
  first
  | rfl; done
  | ring; done
  | fail "Could not prove this goal automatically. Afterall, this is an ad hoc implementation."
)

macro "term_derivation_sub": tactic =>`(tactic|
  first
  | rfl; done
  | ring; done
  | fail "Could not prove this goal automatically. Afterall, this is an ad hoc implementation."
)

macro "term_derivation_product": tactic =>`(tactic|
  first
  | rfl; done
  | ring; done
  | fail "Could not prove this goal automatically. Afterall, this is an ad hoc implementation."
)

macro "term_derivation_div": tactic =>`(tactic|
  first
  | rfl; done
  | ring; done
  | fail "Could not prove this goal automatically. Afterall, this is an ad hoc implementation."
)

macro "term_derivation_finalize" d1:term:1024 d2:term:1024 : tactic =>`(tactic|
  (apply (Iff.mpr $d2); apply (Iff.mp $d1); assumption)
)

macro "term_derivation_chaining_separated_list": tactic =>`(tactic|
  first
  | ring; done
  | (constructor; repeat (intro h; linarith))
  | fail "Could not prove this goal automatically. Afterall, this is an ad hoc implementation."
)

macro "term_derivation_square": tactic =>`(tactic|
  first
  | ring; done
  | fail "Could not prove this goal automatically. Afterall, this is an ad hoc implementation."
)

macro "term_derivation_power": tactic =>`(tactic|
  first
  | rfl; done
  | ring; done
  | fail "Could not prove this goal automatically. Afterall, this is an ad hoc implementation."
)

def h (a b : ℝ) (h1 : a > (0 : ℝ)) (h2 : b > (0 : ℝ)) : a / b + b / a ≥ (2 : ℝ) := by
  have h3 : a > (0 : ℝ) := by old_main_hypothesis
  have h4 : b > (0 : ℝ) := by old_main_hypothesis
  first
  | have h5 : (√ (a / b) - √ (b / a)) ^ 2 ≥ (0 : ℝ) := by calc
    (√ (a / b) - √ (b / a)) ^ 2 = √ (a / b) ^ 2 - (2 : ℝ) * √ (a / b) * √ (b / a) + √ (b / a) ^ 2 := by obvious
    _ = a / b - (2 : ℝ) + b / a := by obvious
    _ ≥ (0 : ℝ) := by obvious
  | have h6 : a / b - (2 : ℝ) + b / a ≥ (0 : ℝ) := by calc
    a / b - (2 : ℝ) + b / a = √ (a / b) ^ 2 - (2 : ℝ) * √ (a / b) * √ (b / a) + √ (b / a) ^ 2 := by obvious
    _ = (√ (a / b) - √ (b / a)) ^ 2 := by obvious
    _ ≥ (0 : ℝ) := by obvious
  have h7 : a / b - (2 : ℝ) + b / a ≥ (0 : ℝ) := by obvious
  have h8 : a / b + b / a ≥ (2 : ℝ) := by
    have d : a = a := by term_derivation_variable
    have d1 : b = b := by term_derivation_variable
    have d2 : a / b = b * a := by term_derivation_div
    have d3 : (2 : ℝ) = 2 := by term_derivation_literal
    have d4 : a / b - (2 : ℝ) = (-2 : ℝ) + b * a := by term_derivation_sub
    have d5 : b = b := by term_derivation_variable
    have d6 : a = a := by term_derivation_variable
    have d7 : b / a = b * a := by term_derivation_div
    have d8 : a / b - (2 : ℝ) + b / a = (-2 : ℝ) + (2 : ℝ) * b * a := by term_derivation_sum
    have d9 : (0 : ℝ) = 0 := by term_derivation_literal
    have d10 : a / b - (2 : ℝ) + b / a ≥ (0 : ℝ) ↔ (-2 : ℝ) + (2 : ℝ) * b * a ≥ (0 : ℝ) := by term_derivation_chaining_separated_list
    have d11 : a = a := by term_derivation_variable
    have d12 : b = b := by term_derivation_variable
    have d13 : a / b = b * a := by term_derivation_div
    have d14 : b = b := by term_derivation_variable
    have d15 : a = a := by term_derivation_variable
    have d16 : b / a = b * a := by term_derivation_div
    have d17 : a / b + b / a = (2 : ℝ) * b * a := by term_derivation_sum
    have d18 : (2 : ℝ) = 2 := by term_derivation_literal
    have d19 : a / b + b / a ≥ (2 : ℝ) ↔ (-2 : ℝ) + (2 : ℝ) * b * a ≥ (0 : ℝ) := by term_derivation_chaining_separated_list
    have d20 : a / b + b / a ≥ (2 : ℝ) := by term_derivation_finalize
    term_derivation_finalize
  obvious

\end{lstlisting}
\end{tcolorbox}


\begin{example}
Problem:
\begin{tcolorbox}[colback=yellow!10, width=\linewidth]
Let $x\in\mathbb{R}$. Let $y\in\mathbb{R}$.
    Prove: $x^2 + y^2 \ge 2xy$.
\end{tcolorbox}

Simplified proof:
\begin{tcolorbox}[colback=blue!10, width=\linewidth]
Trivial since $(x-y)^2 \ge 0$.
\end{tcolorbox}
\end{example}

Elaborated proof:
\begin{tcolorbox}[colback=green!10, width=\linewidth]
Trivial since $(x-y)^2 = x^2 -2xy + y^2 \ge 0$, so $x^2 + y^2 \ge 2xy$.
\end{tcolorbox}

Regularized proof:
\begin{tcolorbox}[colback=red!10, width=\linewidth]
Let $x\in\mathbb{R}$.
Let $y\in\mathbb{R}$.
The goal is to prove $x^2 + y^2 \ge 2xy$.
We have ${{(x-y)}}^2 = x^2 -2xy + y^2 \ge 0$.
We have $x^2 + y^2 \ge 2xy$.
\end{tcolorbox}

Lean 4 code:
\begin{tcolorbox}[colback=white!10, width=\linewidth]
\begin{lstlisting}[language=Lean4]
import Mathlib
syntax "attack" : tactic

macro_rules
| `(tactic| attack) => `(tactic|
  first
  | simp; done
  | ring; done
  | ring_nf; rw [Real.sq_sqrt]; ring; all_goals attack; done
  | nlinarith; done
  | apply sq_nonneg; all_goals attack; done
  | apply div_nonneg; all_goals attack; done
  | field_simp; ring; done
  | linarith; done
)

macro "obvious": tactic =>`(tactic|
  first
  | attack; done
  | congr; all_goals attack; done
  | gcongr; all_goals attack; done
  | fail "Could not prove this goal automatically. It might not be as obvious as you think!"
)

macro "in_set" : term => `(true)



macro "term_trivial": tactic =>`(tactic|
  first
  | simp; done
  | ring; done
  | ring_nf; done
  | linarith; done
  | fail "Could not prove this goal automatically. Afterall, this is an ad hoc implementation."
)

macro "old_main_hypothesis": tactic =>`(tactic|
  first
  | assumption; done
  | fail "Could not prove this goal automatically. Afterall, this is an ad hoc implementation."
)

macro "let_assigned": tactic =>`(tactic|
  first
  | dsimp; done
  | fail "Could not prove this goal automatically. Afterall, this is an ad hoc implementation."
)

macro "term_equivalence": tactic =>`(tactic|
  first
  | simp; done
  | ring; done
  | ring_nf; done
  | linarith; done
  | fail "Could not prove this goal automatically. Afterall, this is an ad hoc implementation."
)

macro "comm_ring": tactic =>`(tactic|
  first
  | ring; done
  | ring_nf; done
  | fail "Could not prove this goal automatically. Afterall, this is an ad hoc implementation."
)

macro "litnum_reduce": tactic =>`(tactic|
  first
  | simp; done
  | simp [*]; done
  | fail "Could not prove this goal automatically. Afterall, this is an ad hoc implementation."
)

macro "litnum_bound": tactic =>`(tactic|
  first
  | linarith; done
  | fail "Could not prove this goal automatically. Afterall, this is an ad hoc implementation."
)

macro "term_derivation_variable": tactic =>`(tactic|
  first
  | rfl; done
  | fail "Could not prove this goal automatically. Afterall, this is an ad hoc implementation."
)

macro "term_derivation_literal": tactic =>`(tactic|
  first
  | rfl; done
  | fail "Could not prove this goal automatically. Afterall, this is an ad hoc implementation."
)

macro "term_derivation_item_path": tactic =>`(tactic|
  first
  | rfl; done
  | fail "Could not prove this goal automatically. Afterall, this is an ad hoc implementation."
)

macro "term_derivation_sum": tactic =>`(tactic|
  first
  | rfl; done
  | ring; done
  | fail "Could not prove this goal automatically. Afterall, this is an ad hoc implementation."
)

macro "term_derivation_sub": tactic =>`(tactic|
  first
  | rfl; done
  | ring; done
  | fail "Could not prove this goal automatically. Afterall, this is an ad hoc implementation."
)

macro "term_derivation_product": tactic =>`(tactic|
  first
  | rfl; done
  | ring; done
  | fail "Could not prove this goal automatically. Afterall, this is an ad hoc implementation."
)

macro "term_derivation_div": tactic =>`(tactic|
  first
  | rfl; done
  | ring; done
  | fail "Could not prove this goal automatically. Afterall, this is an ad hoc implementation."
)

macro "term_derivation_finalize" d1:term:1024 d2:term:1024 : tactic =>`(tactic|
  (apply (Iff.mpr $d2); apply (Iff.mp $d1); assumption)
)

macro "term_derivation_chaining_separated_list": tactic =>`(tactic|
  first
  | ring; done
  | (constructor; repeat (intro h; linarith))
  | fail "Could not prove this goal automatically. Afterall, this is an ad hoc implementation."
)

macro "term_derivation_square": tactic =>`(tactic|
  first
  | ring; done
  | fail "Could not prove this goal automatically. Afterall, this is an ad hoc implementation."
)

macro "term_derivation_power": tactic =>`(tactic|
  first
  | rfl; done
  | ring; done
  | fail "Could not prove this goal automatically. Afterall, this is an ad hoc implementation."
)

def h (x y : ℝ) : x ^ 2 + y ^ 2 ≥ (2 : ℝ) * x * y := by
  first
  | have h1 : (x - y) ^ 2 ≥ (0 : ℝ) := by calc
    (x - y) ^ 2 = x ^ 2 - (2 : ℝ) * x * y + y ^ 2 := by obvious
    _ ≥ (0 : ℝ) := by obvious
  | have h2 : x ^ 2 - (2 : ℝ) * x * y + y ^ 2 ≥ (0 : ℝ) := by calc
    x ^ 2 - (2 : ℝ) * x * y + y ^ 2 = (x - y) ^ 2 := by obvious
    _ ≥ (0 : ℝ) := by obvious
  have h3 : x ^ 2 + y ^ 2 ≥ (2 : ℝ) * x * y := by obvious
  obvious

\end{lstlisting}
\end{tcolorbox}
\end{document}